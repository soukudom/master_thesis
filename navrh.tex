 \section{Architektura}
 Cílem této práce je vytvoření programu, který bude schopen detekovat anomální provoz v IoT sítích. 
 Vzhledem k hardwarovým omezením, které můžou na IoT branách nastat, je architektura navržena tak, aby
 měla co nejmenší nároky na dostupné prostředky. Schéma nasazení detekčního systému se nachází
 na obrázku
 
 Popis device manageru a distributoru, 
 oddělení síťovým interface
 množství možných interfaců
 podrobnější popis v dalších kapitolách
 
 výhoda, že je vlastně jedná o minikompenety, které se mohou nacházet na různých místech
 
 
    \begin{itemize}
      \item popis celkové architektury řešení
      \item lokální a externí detekce
    \end{itemize}
  % celkova architektura reseni, lokalni&externi detekce
  % microservice architektura, mozne rozsireni o IP exporter 

 \section{Kolektor}
 \subsection{Syntaxe konfiguračního souboru}
    
 \section{Detektor}
 \section{Multiplexor a demultiplexor}
 
  \section{Scénáře útoků}
