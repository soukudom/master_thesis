
\section{Integrace kolektoru}    
     Vytvořený kolektor provozních dat je přímou součástí projektu BeeeOn, a~proto jej bylo nutné 
     integrovat do spouštěcího procesu brány. Pro zajištění společné kompilace byly do souborů 
     \textit{CMakeLists.txt} přidány cesty ke zdrojovým souborům kolektoru a závislosti na knihovny
     z~NEMEA frameworku. K~určení způsobu spouštění jednotlivých komponent používá BeeeOn
     soubor \textit{factory.xml}. V~tomto souboru byla vytvořena nová komponenta s~názvem 
     \textit{collector}, 
     která byla následně přidána pod označením \textit{listener} do specifikace ostatních komponent.
     Toto nastavení 
     umožňuje přijímání definovaných událostí v~rámci návrhového vzoru \textit{Observer}.
     Samotná komponenta \textit{collector}
     obsahuje ve svém popisu seznam jmen událostí, kterým přiřazuje pojmenování výstupního \textit{libtrap}
     rozhraní. Formát výstupu je vždy \textit{event-<názevUdálosti>}. 
     
     Vytvořené názvy událostí jsou při spouštění programu pomocí C++ reflexe předány třídě kolektoru
     \textit{NemeaCollector}. Tato třída používá již vytvořené makro \textit{BEEEON\_OBJECT\_TEXT},
     které pro definované 
     typy událostí volá přiřazené členské funkce. Každá událost má členskou funkci s~názvem ve formátu 
     \textit{set<názevUdálosti>} přijímající jeden vstupní parametr typu string s~názvem výstupního 
     rozhraní. V~rámci volání se do instance třídy \textit{EventMetaData}
     nastaví specifické hodnoty členských
     atributů o~daném typu události. Následuje volání členské funkce \textit{initInterface()},
     která přijímá vytvořenou instanci třídy  \textit{EventMetaData} a
     jednotně inicializuje všechny potřebné struktury pro odesílání dat.
     
     Po úspěšném nastavení všech částí se už jen v~rámci návrhového vzoru \textit{Observer} volají
     členské funkce událostí, které jsou definované ve třídě \textit{AbstractCollector} a implementované
     ve třídě \textit{NemeaCollector}. Získané informace z~brány jsou vkládány do UniRec zprávy a odeslány
     výstupním rozhraním.
     
\section{Mux a Demux}    

 Modul \textit{Mux} očekává na vstupu přepínač \textit{-i}, který ve formě řetězce určuje dostupná 
 komunikační rozhraní 
 zajištěné knihovnou \textit{libtrap}. Poslední identifikátor v~zadaném řetězci označuje jméno 
 výstupu. Druhým parametrem je \textit{-n}, který odpovídá počtu vstupních komunikačních
 rozhraní. Při spuštění 
 se pro každý vstup vytvoří samostatné vlákno pomocí knihovny OpenMP. Jelikož je veškerý provoz 
 odesílaný jedním společným výstupem, bylo nutné funkci \textit{trap\_ctx\_send()} vložit do
 kritické sekce, 
 protože pracuje se sdílenými strukturami pro všechny vlákna.
 
 Společné spojení mezi vytvořenými moduly \textit{Mux} a \textit{Demux} používá nastavený typ
 \textit{TRAP\_FMT\_RAW},
 který umožňuje posílat zprávy ve vlastně definovaném formátu. Hlavička vytvořeného formátu
 obsahuje: identifikátor druhu zprávy, číslo komunikačního rozhraní a typ formátu. Obsah přijatých zpráv
 je zapouzdřen do záhlaví. \textit{Mux} při každém přijetí dat kontroluje návratový
 kód funkce \textit{trap\_ctx\_recv()}, který identifikuje nový formát přijatých zpráv. Pokud
 dojde k~detekování změny, tak se pošle \textit{hello} zpráva s~upraveným popisem rozhraní.
 V~ostatních případech se jen přeposílají zapouzdřená data.
 
 Modul \textit{Demux} vyžaduje stejné přepínače jako \textit{Mux}. Jediným rozdílem je, že 
 název společného komunikačního rozhraní, které má \textit{Mux} na posledním místě, musí být zde uveden 
 jako první. Důvodem je, že knihovna \textit{libtrap} zpracovává nejprve vstupní a pak
 výstupní rozhraní. 
 
\section{Zpracování zadaných parametrů}

Načtení konfiguračního souboru má na starosti třída \textit{ConfigParser}, která neprovádí 
žádnou kontrolu vstupních dat, protože se již od návrhu předpokládá, že konfigurace bude 
generována odlišným programem, který zajistí správnost zadaných údajů. Jediný konstruktor třídy
 \textit{ConfigParser} očekává jako vstupní parametr řetězec s~cestou k~cílovému souboru. Během vytváření
 objektu jsou postupně zpracovány jednotlivé řádky zadaného souboru. Zároveň jsou inicializovány
 pole pro uchovávání vypočtených profilů. Jejich délka je specifikována preprocesorovou direktivou
 \textit{\#define DYNAMIC}.
 
Pokud byly zadány parametry pro pravidelný export dat, tak se při inicializaci potřebných struktur
zavolá funkce s~názvem \textit{initExportInterfaces()}, která vytvoří příslušná výstupní komunikační 
rozhraní. Jejich název je vždy vygenerován v~následujícím formátu: 
\textit{u:export-<názevKlíče><idKlíče>}.
   
\section{Výpočet profilu} \label{profileDef}
Veškeré přijaté údaje z~kolektoru jsou v~rámci detektoru ukládány do časových řad. Nad těmito 
daty jsou prováděny výpočty profilů, které využívají detekční funkce. Jelikož jsou profily
přepočítávány s~každým nový prvkem v~časové řadě, tak je důležité provádět výpočty efektivně, 
aby nemusela být vždy procházena celá řady při změně jedné hodnoty. Během implementace byly
vytvořeny členské funkce pro výpočet následujících částí profilu: průměr, klouzavý průměr, klouzavý rozptyl a
klouzavý medián. Tyto členské funkce jsou popsány v~níže uvedených sekcích.

\begin{itemize}
 \item \textbf{getMovingMedian()}%(\textit{sensor\_it, meta\_it, ur\_field, ur\_id})}
 
 
 Klouzavý medián značí hodnotu běžného mediánu odpovídající prvkům umístěných v~časové řadě. 
 Pro jeho výpočet je nutné udržovat položky aktuálního časového okna v~seřazeném pořadí.
 Při vkládání nového prvku je potřeba najít nejstarší prvek v~poli a nahradit ho nově přijatým. 
 Po provedeném nahrazení se pole může nacházet v~neseřazeném stavu a~je 
 vyžadováno jeho opětovné seřazení.
  Tato operace není náročná, protože se změní vždy jen jeden prvek, který je nutné
 přesunou na správné místo. Celková složitost začlenění nového prvku je $O(n)$, protože operace 
 nalezení a seřazení lze provést v~lineárním čase. Paměťové nároky vyžadují alokování dodatečného
 místa, které obsahuje kopii časové řady.
 
 Dále je potřeba zjistit, zda je časová řada sudé nebo liché délky. Pokud je lichá, tak se jako 
 medián vrátí její prvek umístění ve středu. V~opačném případě je výsledkem průměr dvou prostředních
 hodnot.
 
 Aby mohly být provedeny veškeré popsané operace, tak členská funkce pro výpočet mediánu
 potřebuje tyto vstupní parametry:
 \begin{itemize}
   \item \textbf{sensor\_it}: iterátor na časovou řadu daného UniRec pole
   \item \textbf{meta\_it}: iterátor na meta informace o~daném UniRec poli
   \item \textbf{ur\_field}: řetězec s~názvem zpracovávaného UniRec pole
   \item \textbf{ur\_id}: číselný identifikátor přijatého UniRec pole
 \end{itemize}
 
 \item \textbf{getMovingAverageAndVariance()}%(\textit{ur\_field, ur\_id, meta\_it, sensor\_it, meta\_id})}
 
 Tato členská funkce je zaměřená zejména na získání klouzavého rozptylu, ale v~rámci jeho výpočtu je možné
 získat i hodnotu klouzavého průměru, a proto jsou v~návratové hodnotě vraceny oba výsledky.
 Stejně jako v~předchozí sekci klouzavý rozptyl i průměr jsou odvozeny z~aktuálního časového
 okna celé řady.
 Průběh
 zpracování údajů probíhá s~ohledem na vstupní data, která 
 jsou reprezentována stále se posouvajícím časovým oknem. 
 
 Pro umožnění efektivního výpočtu se používá několik pomocným paměťových struktur. První z~nich 
 jsou dvě pole \textit{x} a \textit{x2} se shodnou velikostí jako má hlavní časová řada. Do pole
 \textit{x} jsou ukládána
 nově příchozí data a \textit{x2} udržuje jejich druhé mocniny. Dále je nutné ukládat součty
 hodnot v~těchto polích, které jsou uchovávány v~proměnných \textit{SX} a \textit{SX2}.
 
 V~rámci učící fáze detektoru jsou i zde příchozí prvky postupně vkládány do připravených struktur.
 Pokud jsou již pomocná pole naplněná, tak při příchodu nové položky se používá funkce
 \textit{rotate()}, která provede levou rotaci prvků a umístí nejstarší prvek na konec. Tento
 prvek je poté nahrazen nejnovější hodnotou. Součty jednotlivých polí jsou upraveny podle 
 následujícího předpisu:  
 \[
  součetŘady + novýPrvek - nejstaršíPrvek
\]
 
 Po upravení pomocných struktur je možné provést finální výpočet. Průměr je snadno získán vydělením
 udržovaného součtu počtem prvků v~časovém okně (\textit{N}). Pro určení hodnoty rozptylu se provádí
 následující operace:
\[
   \frac{N * SX2 - SX * SX}{N * (N - 1)}
\]
Pro získání veškerých údajů pro popsané výpočty přijímá členská funkce stejné vstupní parametry
jako \textit{getMedian()} a navíc je ještě očekávána hodnota \textit{meta\_id}, která určuje, 
zda bude v~rámci metadat použita skupina \textit{metaData} nebo \textit{metaProfile} \ref{configParam}. 

 \item \textbf{getOverallAverage()}%(\textit{sensor\_it, meta\_it, meta\_id, ur\_id})}
 
 Poslední možnou částí vypočítávaného profilu sítě je průměr, který je získán v~rámci 
 této členské funkce. Na rozdíl od ostatních metod jsou v~tomto případě zahrnuty i položky 
 mimo časové okno. Výsledná hodnota reprezentuje celkový průměr všech dosud přijatých dat pro 
 daný UniRec klíč. Výpočetní operace využívá znalosti aktuálního průměru a~celkového počtu prvků
 (\textit{M}). Při přijetí 
 nové položky se nová hodnota vytvoří dle následujícího předpisu:
 \[
 \frac{novýPrvek + (M - 1) * aktuálníPrůměr}{M} 
 \]

 Velkou výhodnou je, že nejsou potřeba žádné dodatečné pomocné paměťové struktury a vše je 
 efektivně uloženo do jedné proměnné, která se aktualizuje s~přicházejícími prvky. Vstupní
 parametry jsou totožné s~členskou funkcí \textit{getMovingAverageAndVariance()} až na položku
 {ur\_field}, které zde není potřeba.
\end{itemize}

\section{Detekční metody} \label{detectMethods}
 Na základě vypočtených údajů z~přicházejících dat a podle zapsané konfigurace jsou spouštěny
 členské funkce určené pro detekci incidentů. Popis realizace jednotlivých možností je uveden
 v~následujících sekcích.
 
\begin{itemize}
 \item \textbf{Kontrola limitů}
 
 Tato detekce je reprezentována členskou funkcí \textit{dataLimitCheck()}, která vyhodnocuje
 soft a hard limity. 
 
 V~případě soft limitu se s~každou nově příchozí hodnotou provádí
 porovnání aktuálního profilu sítě s~definovanými hranicemi v~konfiguračním souboru. Pokud dojde
 k~jejich překročení, tak se zvýší hodnota odpovídajícího čítače o~jedna. Jakmile čítač překročí nastavenou 
 hranici pro grace period, tak se odešle zpráva o~detekovaném incidentu. V~případě, že 
 během porovnání hodnota aktuálního profilu nepřekračuje nastavené limity, tak se do čítače uloží 
 číslo 0.
 
 Hard limit provádí stejný způsob porovnání částí aktuálního profilu s~nastavenými hodnotami. Jediným
 rozdílem je, že odeslání incidentu se provede ihned po překročení nakonfigurovaných hranic.
 
 Mezi soft a hard limitem není žádná závislost a lze je používat nezávisle. Při konfiguraci 
 libovolného z~nich se vždy povinně vyžaduje určení minimální a maximální meze současně.
 
 \item \textbf{Kontrola změny}
 
 Neočekávaný nárůst dat je možné detekovat pomocí členské funkce \textit{dataChangeCheck()}.
 Při každém přijetí nové zprávy se provede kontrola poměru mezi hodnotami vzorového a aktuálního
 profilu sítě. V~případě, že vypočítaný poměr překročí nastavené meze, je odeslána informace 
 o~detekované anomálii. Implementace předpokládá, že je vždy nastavena spodní i hodní hranice
 zároveň.
 
 \item \textbf{Periodické kontroly}
 
 V~rámci vytvořeného detektoru jsou implementovány dvě členské funkce s~názvy
 \textit{periodicCheck()} a \textit{periodicExport()}, které provádí požadovanou akci jednou
 za periodu, jejíž délka je určená konfiguračním souborem. Toto chování je realizované pomocí 
 vláken, které je možné uspávat a~probouzet. Vytváření vláken má na starosti členská funkce
 \textit{runThreads()}, která dle nastavených parametrů spouští vlákna pro dostupné UniRec položky.
 
 V~členské funkci \textit{periodicCheck()} jsou prováděny dvě kontroly. První porovnává
 časovou značku posledního přijatého prvku do časové řady s~aktuálním časem. Pokud výsledný rozdíl je větší
 než definovaný limit, tak je odeslána zpráva s~popisem incidentu. Druhá kontrola sleduje hodnoty
 v~časové řadě a pokud s~příchodem nové zprávy nedojde k~její změně, tak je zvýšen čítač o~jedna. 
 Jakmile čítač překročí nastavený limit, tak se odešle zpráva o~nalezené anomálii. V~případě,
 že jsou sousední položky odlišné, tak se čítač nastaví zpět na hodnotu 0.
 
 Druhá členská funkce s~názvem \textit{periodicExport()} zajišťuje pravidelný export položek profilu.
 Vlákno po probuzení pouze naplní připravenou zprávu pro export a pošle ji připraveným výstupním
 rozhraním.
 
\end{itemize}