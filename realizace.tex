
\section{Integrace kolektoru}    
     % popsat trochu jak probihala integrace kolektoru
     Vytvořený kolektor provozních dat je přímou součástí projektu BeeeOn, a proto jej bylo nutné 
     integrovat do spouštěcího procesu brány. Pro zajištění společné kompilace byly do souborů 
     \textit{CMakeLists.txt} přidány cesty ke zdrojovým souborům kolektoru a závislosti na knihovny
     z NEMEA frameworku. K určení způsobu spouštění jednotlivých komponent používá BeeeOn
     soubor \textit{factory.xml}. V tomto souboru byla vytvořena nová komponenta s názvem 
     \textit{collector}, 
     která byla následně přidána pod označením \textit{listener} do ostatních komponent. Toto nastavení 
     umožňuje přijímání definovaných událostí v rámci návrhového vzoru \textit{Observer}.
     Samotná komponenta
     obsahuje ve svém popisu seznam jmen událostí, kterým přiřazuje pojmenování výstupního \textit{libtrap}
     rozhraní. Formát výstupu je vždy \textit{event-<názevUdálosti>}. 
     
     Vytvořené názvy událostí jsou při spouštění programu pomocí C++ reflexe předány třídě kolektoru
     \textit{NemeaCollector}. Tato třída používá již vytvořené makro \textit{BEEEON\_OBJECT\_TEXT},
     které pro definovaný seznam 
     událostí volá příslušné členské funkce. Každá událost má členskou funkci s názvem ve formátu 
     \textit{set<názevUdálosti>} přijímající jeden vstupní parametr typu string s názvem výstupního 
     rozhraní. V rámci volání se do instance třídy \textit{EventMetaData}
     nastaví specifické hodnoty členských
     atributů dané události. Následuje volání členské funkce \textit{initInterface(EventMetaData)},
     která přijímá vytvořenou instanci třídy  \textit{EventMetaData} a
     jednotně inicializuje všechny potřebné struktury pro odesílání dat.
     
     Po úspěšném nastavení všech částí se už jen v rámci návrhového vzoru \textit{Observer} volají
     členské funkce událostí, které jsou definované ve třídě \textit{AbstractCollector} a implementované
     ve třídě \textit{NemeaCollector}. Získané informace jsou vkládány do UniRec zprávy a odeslány
     výstupním rozhraním.
     
\section{Mux a Demux}    

 Modul \textit{Mux} očekává na vstupu přepínač \textit{-i}, který ve formě řetězce určuje dostupná 
 rozhraní 
 zajištěné knihovnou \textit{libtrap}. Poslední identifikátor v řetězci označuje jméno 
 výstupu. Druhým parametrem je \textit{-n}, který odpovídá počtu vstupních rozhraní. Při spuštění 
 se pro každý vstup pomocí knihovny OpenMP vytvoří samostatné vlákno. Jelikož je veškerý provoz 
 odesílaný jedním společným rozhraním, bylo nutné funkci \textit{trap\_ctx\_send} vložit do
 kritické sekce, 
 protože pracuje se sdílenými strukturami pro všechny vlákna.
 
 Společné spojení mezi moduly vytvořenými \textit{Mux} a \textit{Demux} používá nastavený typ
 \textit{TRAP\_FMT\_RAW},
 který umožňuje posílat zprávy ve vlastně definovaném formátu. Hlavička vytvořeného formátu
 obsahuje: identifikátor druhu zprávy, číslo rozhraní a typ formátu. Obsah přijatých zpráv
 je zapouzdřen do záhlaví. \textit{Mux} při každém přijetí dat kontroluje návratový
 kód funkce \textit{trap\_ctx\_recv}, který identifikuje nový formát přijatých zpráv. Pokud
 dojde k detekování změny, tak se pošle \textit{hello} zpráva s upraveným popisem rozhraní.
 V ostatních případech se jen přeposílají zapouzdřená data.
 
 Modul \textit{Demux} vyžaduje stejné přepínače jako \textit{Mux}. Jediným rozdílem je, že 
 název společného rozhraní, které má \textit{Mux} na posledním místě, musí být zde uveden 
 jako první. Důvodem je, že knihovna \textit{libtrap} zpracovává nejprve vstupní a pak
 výstupní rozhraní. 
 
\section{Zpracování zadaných parametrů}

Načtení konfiguračního souboru má na starosi třída \textit{ConfigParser}, která neprovádí 
žádnou kontrolu vstupních dat, protože se již od návrhu předpokládá, že konfigurace bude 
generována odlišným programem, který zajistí správnost parametrů. Jediný konstruktor třídy
 \textit{ConfigParser} očekává jako parametr řetězec s cestu k cílovému souboru. Během vytváření
 objektu jsou postupně zpracovány jednotlivé řádky zadaného souboru. Zároveň budou incializovány
 pole pro uchovávání vypočtených profilů. Jejich délka je specifikována preprocesorovou direktivou
 \textit{\#define DYNAMIC}.
 
Pokud byly zadány parametry pro pravidelný export dat, tak při inicializaci potřebných struktur
zavolá funkce s názvem \textit{initExportInterfaces}, která vytvoří příslušná výstupní 
rozhraní. Jejich název je vždy vygenerován v následujícím formátu: 
\textit{u:export-<názevKlíče><idKlíče>}.
   
\section{Výpočet profilu}   
   %popis vypoctu profilu
\section{Detekční funkce} 
 % algoritmus detektoru popsat jednotlivé detekční metody
  % popsat, ze periodicky veci jsou delany pomoci vlaken per polozku
    % hlavní je že je to proto, že se nedozvi info o čase když nepřijdou data, tak to musim dělat sám