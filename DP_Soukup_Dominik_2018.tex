% options:
% thesis=B bachelor's thesis
% thesis=M master's thesis
% czech thesis in Czech language
% slovak thesis in Slovak language
% english thesis in English language
% hidelinks remove colour boxes around hyperlinks

\documentclass[thesis=M,czech]{FITthesis}[2012/06/26]

\usepackage[utf8]{inputenc} % LaTeX source encoded as UTF-8

\usepackage{graphicx} %graphics files inclusion
% \usepackage{amsmath} %advanced maths
% \usepackage{amssymb} %additional math symbols

\usepackage{dirtree} %directory tree visualisation
\usepackage{makecell}

\renewcommand\theadalign{cb}
\renewcommand\theadfont{\bfseries}
\newcolumntype{C}[1]{>{\centering\arraybackslash}m{#1}}

% % list of acronyms
% \usepackage[acronym,nonumberlist,toc,numberedsection=autolabel]{glossaries}
% \iflanguage{czech}{\renewcommand*{\acronymname}{Seznam pou{\v z}it{\' y}ch zkratek}}{}
% \makeglossaries

\newcommand{\tg}{\mathop{\mathrm{tg}}} %cesky tangens
\newcommand{\cotg}{\mathop{\mathrm{cotg}}} %cesky cotangens

% % % % % % % % % % % % % % % % % % % % % % % % % % % % % % 
% ODTUD DAL VSE ZMENTE
% % % % % % % % % % % % % % % % % % % % % % % % % % % % % % 

\department{Katedra počítačových systémů}
\title{Detekce anomálií v~provozu IoT sítí}
\authorGN{Dominik} %(křestní) jméno (jména) autora
\authorFN{Soukup} %příjmení autora
\authorWithDegrees{Bc. Dominik Soukup} %jméno autora včetně současných akademických titulů
\author{Dominik Soukup} %jméno autora bez akademických titulů
\supervisor{Tomáš Čejka}
\acknowledgements{Doplňte, máte-li komu a za co děkovat. V~opačném případě úplně odstraňte tento příkaz.}
\abstractCS{V~několika větách shrňte obsah a přínos této práce v~češtině. Po přečtení abstraktu by se čtenář měl mít čtenář dost informací pro rozhodnutí, zda chce Vaši práci číst.}
\abstractEN{Sem doplňte ekvivalent abstraktu Vaší práce v~angličtině.}
\placeForDeclarationOfAuthenticity{V~Praze}
\declarationOfAuthenticityOption{4} %volba Prohlášení (číslo 1-6)
\keywordsCS{Nahraďte seznamem klíčových slov v~češtině oddělených čárkou.}
\keywordsEN{Nahraďte seznamem klíčových slov v~angličtině oddělených čárkou.}
% \website{http://site.example/thesis} %volitelná URL práce, objeví se v tiráži - úplně odstraňte, nemáte-li URL práce

\begin{document}

% \newacronym{CVUT}{{\v C}VUT}{{\v C}esk{\' e} vysok{\' e} u{\v c}en{\' i} technick{\' e} v Praze}
% \newacronym{FIT}{FIT}{Fakulta informa{\v c}n{\' i}ch technologi{\' i}}

\begin{introduction}
Koncept internetu existuje již několik desítek let a pro spoustu lidí se stal 
nedílnou součástí pracovního i osobního života. V~poslední době je možné sledovat
stále rostoucí počet zařízení, která jsou do něj zapojena. Tento trend by měl
pokračovat i do budoucnosti, a dokonce v~ještě větším měřítku. Odhadem je 
přes 30 miliard připojených zařízení do roku 2020 \cite{iotDevices}.
Důvodem zrychleného
růstu je expanze síťového připojení na veškeré elektronické zařízení a senzory, 
které umožní vzdálené ovládání a monitorování. Pro označení tohoto trendu se používá
termín Internet věcí (Internet of Things, IoT).

Cílem IoT je usnadnit, zlepšit a ušetřit lidskou činnost napříč všemi odvětvími. 
Uplatnění se nachází zejména ve výrobních podnicích, dopravě nebo běžných domácnostech.
Příklad konkrétního nasazení popisuje článek \cite{sleeping}, jehož cílem je měření 
kvality spánku a odhalení případných poruch. Monitorovací systém lze dále 
propojit například s~ovládáním místnosti, které bude reagovat na aktuální úroveň spánku
úpravou světel, oken nebo vzduchu.

Dále je upraven model komunikace, který již nevyžaduje zasílání zpráv centrálnímu 
serveru (north-south), ale podporuje přímou komunikaci mezi připojenými uzly (east-west).
Oblast IoT nezahrnuje jen malá a nevýkonná zařízení, ale jeho součástí jsou i 
výkonná datová centra a pokročilé algoritmy, které vyhodnocují získané informace.

Hlavní hrozbou Internetu věcí je zejména bezpečnost a ochrana soukromí. Dochází zde k~přenosu
citlivých dat, která slouží
k~automatizovanému řízení dalších
systémů, monitorování prostředí a zabezpečovacím účelům. Zároveň s~masivním rozšířením nově
připojených zařízení roste riziko vzniku nových útoků a možnosti způsobení větších
škod. Příkladem bezpečnostního incidentu je útok na distribuční síť elektrického 
proudu na Ukrajině, který měl dopad na 225 000 zákazníků \cite{ukraine}. 

Pro potlačení možného vzniku hrozeb musí být součástí každé dnešní IoT sítě sada procesů,
které umožní důvěryhodné
získávání validních dat, vzdálenou správu a možnost
detekce anomálií jako v~běžných IP (Internet Protocol) sítích. 	
\end{introduction}

\chapter{Cíl práce}
Cílem diplomové práce je analyzovat množinu aktuálně používaných protokolů
pro komunikaci v~IoT sítích a identifikovat jejich bezpečnostní zranitelnosti.
Při analýze bude věnována pozornost zejména bezdrátovým senzorovým protokolům.
Na základě získaných znalostí bude navržen, implementován a otestován algoritmus
pro monitorování a automatickou detekci anomálního provozu v~IoT sítích.
Algoritmus bude možné spustit
v~prostředí nově vznikající opensource brány BeeeOn \cite{beeeon}, čímž dojde k~rozšíření 
dostupných bezpečnostních funkcí.


\chapter{Analýza}
Kapitola se zabývá analýzou celkové architektury IoT sítí a způsoby pro její
zabezpečení. Postupně je prozkoumán komunikační model, možné bezpečnostní hrozby 
a existující řešení pro obranu. Na základě analýzy jsou uvedeny funkční a nefunkční
požadavky, které jsou kladeny na výsledný program. V~závěru jsou vybrány konkrétní
technologie pro realizaci.

 \section{Architektura IoT sítí}
  V~blízké době se očekává stále větší nárůst zařízení, která jsou připojena k~internetu.
  Dle odhadů by jejich počet měl v~roce 2020 překročit 30 miliard \cite{iotDevices}.
  Pro takové množství připojení už není možné, aby každé zařízení komunikovalo přímo
  se vzdáleným datovým centrem, protože nároky na potřebnou šířku pásma by byly 
  obrovské \cite{fog}.
  Dalším problémem je často velmi omezený výkon připojených prvků, který je nezbytný pro 
  použití bezpečnostních funkcí umožňujících kompletně zabezpečenou komunikaci. 
  
  Řešení těchto problémů je do probíhající komunikace přidat několik podvrstev, 
  které umožní přesunout výpočetní výkon blíže ke koncovým zařízením, a tím celý
  proces zpracování dat provést efektivněji.
  
  V~následujících podkapitolách bude popsán nový model komunikace, který přináší 
  IoT sítě.
 \subsection{Fog computing} 
 Fog computing je rozšíření cloud computingu, které spočívá v~přesunutí výpočetního
 výkonu blíže k~okraji sítě. Rozšíření je umožněno pomocí přidání síťových zařízení,
 které kromě běžných funkcionalit poskytují i výpočetní výkon pro běh externích programů. Programy 
 je často možné nasadit pomocí kontejnerů nebo samostatných virtuálních strojů, což 
 velmi usnadňuje jejich distribuci \cite{fog}.
 
 Porovnání klasické a fog architektury z~hlediska zpracování dat se nachází
 na Obrázku \ref{obr.fog}. V~reálném nasazení může
 být použito i více fog vrstev, kde každá provádí určitý stupeň předzpracování a řízení
 dat.
\begin{figure}[ht]
\begin{center}
\includegraphics[scale=0.45]{pictures/fog-arch}
\caption{Porovnání klasické a fog architektury}
\label{obr.fog}
\end{center}
\end{figure}
 Zavedením principů fog computingu vznikají pro síť následující výhody:
 \begin{itemize}
 \item \textbf{Zlepšení bezpečnosti} 
 
     Síťové prvky jsou trvale napájené a připojené k internetu. Podporují pokročilé
     bezpečnostní funkce, a proto je možné například vytvářet šifrované tunelové
     spojení pro bezpečný přenos dat \cite{fog}.     
\item \textbf{Nižší nároky na šířku pásma a latenci} 

    Odeslaná data z~koncových zařízení jsou zpracovávána a filtrována na okraji 
    sítě. Tím je možné rychleji reagovat na přijaté zprávy a snížit nároky
    na latenci a šířku pásma. Zároveň krátkodobá data mohou být uložené ve 
    fog vrstvě a centrální datové centrum může být využito pro dlouhodobé údaje, 
    které se zpracovávají pokročilými algoritmy pro analýzu dat \cite{fog}.
 \item \textbf{Jednotná správa} 
 
    Při správě sítě už se nemusí přistupovat přímo na koncové prvky, které 
    často komunikují různými protokoly, ale stačí pouze
    řídit síťová zařízení v~jednotlivých fog vrstvách, které odstiňují různorodost
    protokolů a nabízí standardizovaný přístup. Díky této abstrakci je 
    zároveň zjednodušeno zpracování získaných dat a je umožněno přímé zasílání zpráv
    mezi koncovými prvky, které používají odlišné komunikační protokoly \cite{fog}.
    
    Další výhodou je, že výpočetní i síťové funkce jsou zajištěny jedním hardwarovým
    zařízením, což značně usnadňuje požadavky pro nasazení.
 \end{itemize} 
 
 Nevýhodou naopak může být distribuovaná topologie, která má větší nároky na údržbu.
 
 \subsection{IoT brána} 
 IoT brána je síťové zařízení, které je umístěno velmi blízko koncových zařízení
 a představuje vstup do fog vrstvy. Jejím hlavním cílem je získávat data 
 z~připojených zařízení a poskytovat je vyšším vrstvám. Pokud je brána reprezentována
 výkonnějším síťovým prvkem, tak v~něm zároveň může probíhat i základní zpracování
 dat. 
 
 Pro IoT sítě je typické, že obsahují velké množství koncových prvků komunikujících
 různorodými způsoby. Zejména senzory používají protokoly, které nepodporují
 IP spojení. Důvodem použití této komunikace je často velký
 důraz na nízkou spotřebu a specifické požadavky na způsob zasílání zpráv.
 Příkladem protokolů pro senzorové sítě je například:
 Z-Wave, Bluetooth a Zigbee. Jejich detailní popis se nachází v~sekci 
 \ref{protokoly}.
 Tato různorodost vyžaduje, aby brána obsahovala dodatečná komunikační rozhraní, které 
 umožní připojení nejrůznějších bezdrátových i drátových koncových prvků. 
 
 V~současné době existuje mnoho různých bran jejichž parametry se liší dle 
 způsobu nasazení a provozních nároků. Velkým problémem v~této oblasti je malý 
 důraz na bezpečnostní funkce, které umožní vzdálené řízení brány, kontrolu provozu a 
 aktualizace programového vybavení. Z~těchto důvodů vznikl opensource projekt BeeeOn \cite{beeeon}
 jehož cílem je vytvořit softwarovou IoT bránu, kterou bude možné spustit na různých 
 hardwarových platformách. BeeeOn brána je navržena modulárně tak, aby byla schopna 
 zpracovávat více rozdílných senzorových protokolů, a tím bude možné provozovat jedno
 univerzální zařízení namísto několika proprietárních. Zároveň je kladen důraz na bezpečnost, 
 a proto veškeré údaje, které je možné získat o~provozu, jsou poskytovány pro analýzu. Nad těmito údaji
 je postaven detekční algoritmus, který je výsledkem této diplomové práce.
 

 \subsection{Komunikační model a jeho hrozby}
 Při použití principů popsaných v~předchozích kapitolách lze model komunikace rozdělit
 do následujících vrstev \cite{iotSurvey}:
 \begin{itemize}

 \item \textbf{Senzorová vrstva}
 
 Senzorová vrstva obsahuje veškerá koncová zařízení, které získávají informace ze svého
 okolí nebo vykonávají potřebnou službu \cite{secFramework}. Tato zařízení jsou připojena
 kabelově nebo bezdrátově k~IoT bráně. K~jedné bráně může být připojeno několik
 prvků, které komunikují odlišnými způsoby. Princip komunikace a možnosti topologie se odlišují
 na základě použité technologie.
 
 Velkým nebezpečím této vrstvy jsou zejména bezdrátové protokoly, protože při nepoužití
 zabezpečení může snadno dojít k~odposlouchávání nebo úpravám provozu \cite{iotSurvey}.
 Dále se zde mohou vyskytovat zařízení, které jsou označeny jako zabezpečené, 
 ale díky starší verzi komunikačního protokolu používají zastaralé bezpečnostní funkce
 nebo obsahují implementační chyby. 
 Tento případ je velmi nebezpečný, protože vyvolává falešný pocit bezpečí. Útoky se také mohou 
 zaměřovat na prvky s~bateriovými zdroji, které mohou být nadměrnou komunikací
 účelově vybíjeny.
 
 \item \textbf{Síťová vrstva}
 
  Po zpracování senzorových dat na bráně je nutné získané informace odeslat dalším
 službám. K~tomuto účelu slouží síťová vrstva. Cílem této vrstvy je také umožnit 
 vzdálenou správu brány \cite{secFramework}. Pro výběr konkrétního protokolu je 
 nutné znát rozhraní aplikační vrstvy. Ve většině případů je  spojení vytvořeno
 pomocí protokolu HTTPS (Hypertext Transfer Protocol Secure) nebo technologie
 VPN (Virtual Private Network). Nad tímto spojením je postavena další služba pro 
 výměnu zpráv. Příkladem může být: MQTT (Message Queuing Telemetry Transport),
 CoAP (Constrained Application Protocol) nebo
 AMQP (Advanced Message Queuing Protocol). Tyto protokoly jsou podrobněji popsány
v~kapitole \ref{protokoly}.
 
 Bezpečnostní hrozby síťové vrstvy jsou stejné jako v~klasických sítích. Je potřeba
 dodržet principy důvěry, integrity a dostupnosti. Tímto přístupem je možné
 předejít útokům jako: DDoS (Distributed Denial Of Service),
 MITM (Man In The Middle) a podvržení informací. Zároveň je nutné
 nezapomenout, že se zde většinou vyskytuje M2M (Machine To Machine)
 komunikace a je důležité použít 
 vhodná komunikační rozhraní \cite{iotSurvey}. Častým případem bývá zastaralá 
 verze firmware, jehož napadení může vést k~nestandardnímu chování nebo dokonce 
 ke kompletnímu převzetí kontroly. 
 
 \newpage
 \item \textbf{Aplikační vrstva}
 
 Aplikační vrstva se stará o~ukládání dlouhodobých dat a jejich finální zpracování. 
 Zároveň zobrazuje uživateli zpracované informace a umožňuje provádět konfiguraci
 celé sítě. Z~důvodu její možné rozsáhlosti je kritické, aby správa
 topologie podporovala automatizaci. \cite{secFramework}.
 
 Tato vrstva je umístěna vetšinou v~datovém centru a umožňuje vzdálený přístup. 
 Její bezpečnostní problémy lze přirovnat k~problémům cloud computingu. Dle
 množství požadovaných funkcí může být různě složitá a s~rostoucí složitostí
 se také liší nároky na úroveň zabezpečení. Příkladem možných útoků může být:
 Buffer Overflow, SQL Injection nebo DDoS.
 
\end{itemize}

Mezi obecné problémy patří způsob nasazení. Koncové IoT prvky mají odlišné parametry než 
běžná uživatelské zařízení. Zároveň se liší i průběh komunikace, která je v~IoT méně heterogenní
a často založena na M2M. V~případě umístění všech koncových prvků do jednoho společného segmentu
se komplikuje nastavení bezpečnostních pravidel a také roste možný dopad provedeného útoku. 
Při úspěšném napadení jednoho prvku útočník získá možnost rozšíření na další uzly ve stejném
segmentu.

 
 \newpage
 
 \section{Používané komunikační protokoly} \label{protokoly}
 Jedním z~hlavních cílů IoT je možnost vzájemného propojení různých komunikačních protokolů, 
 které umožní automatizovanou výměnu zpráv mezi všemi dostupnými zařízeními. Tímto způsobem
 je následně možné zefektivňovat a usnadňovat lidskou práci.
 
 V~následujících podkapitolách bude popsána množina aktuálně často používaných protokolů
 včetně jejich bezpečnostních funkcí.
 
  \subsection{MQTT}
  MQTT je otevřený síťový komunikační protokol typu publish/subscribe, který byl navržen
  v~roce 1999. Již od návrhu byl zaměřen na nízkou náročnost komunikace a jednoduchost
  implementace. Díky těmto vlastnostem je velmi vhodný pro IoT a M2M systémy \cite{mqtt}.
  \subsubsection{Způsob komunikace}
  
  Protokol MQTT je postaven nad transportní vrstvou TCP (Transmission Control Protocol)
  a využívá model publish/subscribe, který vychází z~tradičního způsobu zasílání zpráv
  typu klient-server. Roli serveru zde plní speciální uzel, který se nazývá broker.
  Broker je známý všem ostatním klientům, kteří mohou zasílat zprávy pomocí operace
  \textit{publish} nebo se přihlásit o~příjem zpráv díky operaci \textit{subscribe}.
  Na základě provedených operací broker přijímá zprávy a rozhoduje o~jejich přeposlání.
  Způsob odeslání zprávy závisí na obsažených metadatech.
  
  Nejčastěji o~směru odeslání rozhoduje předmět (topic) zprávy. Předmět je tvořen
  jednoduchým UTF-8 řetězcem s~hierarchickou strukturou, ve které jsou jednotlivé
  vrstvy odděleny lomítkem a každá z~nich musí obsahovat minimálně jeden
  znak (např. domov/přízemí/světloŠatna). V~předmětu
  zprávy mohou být některé vrstvy nahrazeny zástupnými symboly + a \#. Symbol +~dokáže
  nahradit pouze jednu úroveň předmětu libovolným řetězcem a symbol \# umožňuje
  zastoupit více úrovní. Díky těmto symbolům mohou klienti odeslat nebo přijímat
  zprávy z~více témat. Schéma topologie a ukázka využití předmětů zpráv se nachází na 
  Obrázku \ref{obr.mqtt-arch}.
  
  \begin{figure}[ht]
\begin{center}
\includegraphics[scale=0.41]{pictures/mqtt-arch}
\caption{MQTT architektura}
\label{obr.mqtt-arch}
\end{center}
\end{figure}
  
  Další užitečnou položkou protokolu MQTT je QoS (Quality of Service),
  která může nabývat hodnot 0, 1 nebo 2.
  
   \begin{itemize}
    \item \textbf{QoS 0}
    
    Veškeré zprávy jsou odesílány bez potvrzení a žádným způsobem
    není zvýšena úroveň spolehlivosti, která je shodná se spolehlivostí protokolu TCP.
    
    \newpage
    \item \textbf{QoS 1}
    
    Pomocí potvrzování zajišťuje, že každá zpráva bude příjemci doručena alespoň jednou.
    
    \item \textbf{QoS 2}
    
     Umožňuje, aby každá zpráva byla spolehlivě doručena právě jednou.
   \end{itemize}
  Hodnota QoS se nastavuje vždy mezi dvěma uzly při navazování spojení.
  Z~pohledu brokeru se může stát, že přijatá a odeslaná zpráva mají jiné QoS.
  Úrovně 1 a 2 dále umožňují perzistentní ukládání zpráv v~případě, že příjemce je
  nedostupný. Zároveň platí, že s~vyšší úrovní roste i režie komunikace \cite{mqtt_intro}.
 \subsubsection{Bezpečnost}
 Zabezpečení protokolu MQTT je možné rozdělit do následujících vrstev:
 \begin{itemize}
  \item \textbf{Síťová vrstva}
  
    Veškerá komunikace je postavena nad TCP/IP, a proto lze probíhající komunikaci
    zapouzdřit pomocí VPN připojení
    jako v~běžných počítačových sítích. Kvůli větším nárokům na výkon je toto
    řešení vhodnější pro výkonnější zařízení jako jsou například IoT brány, které
    mohou s~brokerem navázat site-to-site spojení \cite{mqtt_sec}.
    
  \item \textbf{Transportní vrstva}
  
  Na této úrovni se využívá šifrování provozu pomocí protokolu TLS (Tra\-nsport
  Layer Security). Omezením této metody jsou požadavky na výkon, které mohou být
  poměrně vysoké pokud nastává časté navazování spojení \cite{mqtt_sec}.
  
  \item \textbf{Aplikační vrstva}
  
  Samotný protokol MQTT nedefinuje žádné šifrovací mechanismy na aplikační úrovni. 
  Zabezpečení dat zde musí zajistit uživatel ještě před jejich zapouzdřením do MQTT zprávy. 
  Ovšem tímto způsobem je možné šifrovat jen tělo zprávy a hlavička zůstává nezměněná.
  
  Pro autentizaci je možné využít ověření pomocí jména a hesla nebo x.509 certifikátu.
  Jméno a heslo je přenášeno nešifrovaně, a proto je vhodné tuto metodu doplnit se 
  zabezpečením síťové nebo transportní vrstvy. Autentizaci pomocí certifikátů 
  je možné využít jen v~případě použití TLS. Tato metoda je vhodnější pokud všechna
  zařízení jsou po jednotnou správou a je možné automatizovat distribuci 
  klientských certifikátů.
  
  Dále je na straně brokeru možné definovat pravidla pro autorizaci. Tato pravidla 
  přiřazují klientů oprávnění pro provedení operací publish a subscribe
  nad příslušnými tématy \cite{mqtt_sec}.
  
 \end{itemize}

  \subsection{CoAP}
  CoAP je otevřený přenosový protokol určený pro
  komunikaci síťových zařízení s~velmi omezeným výkonem. Návrh vychází z~RESTful
  (Representational State Transfer) principů, tudíž jeho použití je velmi vhodné
  v~prostředích s~již existujícím webovým rozhraním, do kterého se snadno integruje. \cite{coap}
  
   \subsubsection{Způsob komunikace}
   Komunikace je postavena nad protokolem UDP (User Datagram Protocol) a vychází
   z~modelu request/response, který se využívá u~protokolu HTTP (Hypertext
   Transfer Protocol). Oproti protokolu HTTP je zde omezen počet možných operací
   a komunikace probíhá asynchronně. Dle důležitosti odesílané zprávy je možné určit,
   zda se má zasílat potvrzení či nikoli. Protože veškerá komunikace probíhá nad
   protokolem UDP, tak přenášené zprávy obsahují položku Message ID, která dokáže
   ošetřit duplikaci přijatých dat.
   
   Pro zajištění integrace s~běžnými webovými službami a zachování nízkých nároků
   se v~CoAP topologii velmi často vyskytují proxy servery. Tyto servery mohou plnit
   funkci běžných reverzních a dopředných proxy, mapování mezi CoAP a HTTP protokolem
   nebo vyvažování zátěže. Způsob komunikace a schéma možné způsoby propojení jsou na
   Obrázku \ref{obr.coap-arch}. Senzor reprezentuje koncový prvek s~omezeným výkonem.
   
   \begin{figure}[ht]
   \begin{center}
   \includegraphics[scale=0.41]{pictures/coap-arch}
   \caption{CoAP architektura}
   \label{obr.coap-arch}
   \end{center}
   \end{figure}
   
   Ve specifikaci protokolu CoAP jsou definovány následující operace:
   \begin{itemize}
    \item \textbf{GET}
    
    Metoda \textit{GET} vrací aktuální stav požadovaného zdroje, který je identifikován pomocí
    URI (Uniform resource identifier). Tato operace je vždy bezpečná a idempotentní.
    
    \item \textbf{POST}
    
    \textit{POST} požadavek obsahuje ve svém těle novou reprezentaci cílového zdroje a požaduje 
    jeho zpracování. Funkce, která
    novou reprezentaci přijímá je definovaná na cílovém uzlu s~příslušným URI.
    Výsledkem je vytvoření nového zdroje nebo aktualizace původního. Tato metoda
    není z~pohledu zpracování bezpečná ani idempotentní.
    
    \item \textbf{PUT}
    
    Tato metoda specifikuje nový stav cílového zdroje, který specifikován použitým URI. Zdroj
    se buď vytvoří, nebo 
    v~případě jeho existence aktualizuje. Provedení operace není bezpečné, ale 
    je idempotentní.
    
    \item \textbf{DELETE}
    
    Operace \textit{DELETE} požaduje smazání zdroje s~příslušným URI. Tato metoda není bezpečná,
    ale je idempotentní.
    
   \end{itemize}
   
   Vyhledání potřebného zdroje je v~topologii protokolu CoAP možné provést pomocí
   znalosti jeho URI nebo odesláním multicastového požadavku na definovanou
   skupinu uzlů. Možnost vyhledávání cílových zdrojů je velmi důležitá zejména
   v~M2M prostředích \cite{coap}.
   
   \subsubsection{Bezpečnost}
   Samotný protokol nijak nedefinuje možnost autentizace a autorizace. V~případě
   potřeby je nutné tyto mechanismy implementovat v~aplikačním kódu. 
   
   Pro zajištění šifrování provozu nabízí CoAP následující režimy:
   \begin{itemize}
    \item \textbf{NoSec}
    
    Tento mód neobsahuje žádnou úroveň zabezpečení a veškeré zprávy jsou zasílány
    v~otevřené podobě. 
    
    \item \textbf{PreSharedKey}
    
    V~tomto případě se naváže spojení pomocí protokolu DTLS (Datagram Transport
    Layer Security), které využívá symetrické šifrování. Předsdílený klíč musí
    být známý všem uzlům před zahájením komunikace. 
    
    \item \textbf{RawPublicKey}
    
    Tento režim také navazuje zabezpečené spojení pomocí protokolu DTLS, ale místo
    symetrické šifry se používá asymetrická.
    
    \item \textbf{Certificate}
    
    Mód \textit{Certificate} rozšiřuje \textit{RawPublicKey} o~přidání certifikátu.
   \end{itemize}

   Podobně jako u~protokolu MQTT je i zde možné ochránit provoz na síťové
   vrstvě pomocí VPN. Při nasazení libovolného režimu zabezpečení je nutné počítat 
   s~většími nároky na výkon, které musí klient splňovat pro navazování a udržování 
   spojení \cite{coap}.
   
  \subsection{Z-Wave}
  
  Z-Wave je bezdrátový komunikační protokol určený pro senzorové sítě, který vysílá 
  v~subgigahercových pásmech ISM (Industrial, Scientific and Medical). Veškeré komunikační
  prvky jsou certifikovány aliancí Z-Wave, která zároveň poskytuje technickou dokumentaci a
  licence pro vývoj. V~současné době existují certifikace Z-Wave a Z-Wave Plus. 
  Z-Wave Plus zařízení obsahují nový chipset, který vylepšuje komunikační parametry
  sítě a zároveň je zpětně kompatibilní se staršími modely \cite{z-plus}.
  Otevřená implementace celého protokolu se nazývá OpenZWave \cite{openzwave}, která
  je vyvíjena komunitou \cite{cesnet-survey}.
 
 \subsubsection{Způsob komunikace}
 V~Z-Wave síti se může maximálně vyskytovat 232 uzlů, mezi kterými se vždy nachází jeden označený
 jako kontroler. Pro přidání libovolného zařízení do sítě musí být nejprve provedeno přímé spárování
 s~kontrolerem. Během párovacího procesu nový prvek získá vlastní 8 bitový identifikátor (\textit{Node ID}) a 
 unikátní 32 bitový identifikátor sítě (\textit{Home ID}), který má již od výroby kontroler uložen v~nepřepisovatelné paměti.
 Následné zasílání zpráv už nemusí probíhat přímo mezi senzorem a kontrolerem, ale zprávy mohou
 být přeposílány sousedními prvky, které jsou trvale napájené, čímž se velmi zvyšuje možná rozloha sítě. 
 Pro odebrání libovolného zařízení je opět nutné zajistit přímé spojení s~kontrolerem a spustit odstraňovací
 proces \cite{cesnet-survey}.
 
 V~rámci specifikace Z-Wave \cite{zwave-spec} je definován mechanismus pro určení dostupných příkazů. Každý senzor obsahuje
 svou definici tříd funkcionalit (\textit{Command Class}), které udržují dostupné příkazy a formát odpovědi. Tyto definice
 jsou předány kontroleru během procesu párování. Příkladem může být třída \textit{Binary Switch}, která má k~dispozici příkazy:
 \begin{itemize}
 \item \textit{SET} -- odesílá kontroler pro nastavení hodnoty 
 \item \textit{GET} -- odesílá kontroler pro získání hodnoty  
 \item \textit{REPORT} -- odesílá senzor jako odpověď na dotaz \textit{GET} 
 \end{itemize} 
 
 Při posílání zpráv je vždy vyžadováno potvrzení. V~případě neobdržení potvrzení se vysílání opakuje. 
 Po třetím neúspěšném pokusu je požadavek zahozen.
 
 \subsubsection{Bezpečnost}
 Původní verze protokolu Z-Wave umožňovala volitelně využívat šifrování pomocí 128 bitového AES (Advanced Encryption Standard).
 Výměna symetrického klíče v~této verzi probíhá během počátečního párování, kdy je šifrován pomocí
 výchozího
 klíče, který je uložen ve firmwaru. Vzhledem k~tomuto postupu je dobré provádět počáteční párování na bezpečném
 místě, kde nemůže dojít k~odposlechu \cite{zwave-S0-attack}.
 
 V~roce 2016 Z-Wave aliance vydala nový S2 (Security 2) framework, který vylepšuje bezpečnostní funkcionality 
 a od roku 2017 je jeho použití povinné pro všechny nově certifikovaná zařízení. 
 
 S2 framework umožňuje zařízení rozdělit do následujících skupin s~rozdílnými šifrovacími klíči:
 \begin{itemize}
  \item \textbf{Access Control} -- 
   nejdůvěryhodnější třída, která obsahuje bezpečnostní prvky jako jsou například zámky,
   které zároveň podporují autentizaci
  \item \textbf{Authenticated} -- 
  skupina určená pro běžné senzory, které podporují autentizaci 
  \item \textbf{Unauthenticated} -- 
  třída pro ostatní prvky, které nepodporují autentizaci.
 \end{itemize}
 Autentizace probíhá pomocí PIN (Personal Identification Number) nebo QR (Quick Response) kódu.
 Výměna klíče je založena na algoritmu ECDH (Elliptic-curve Diffie–Hellman), který zajišťuje 
 dostatečnou úroveň bezpečnosti během párovacího procesu \cite{cesnet-survey}.

 
  \subsection{BLE (Bluetooth Low Energy)} 
  BLE je bezdrátový protokol určený pro senzorové sítě s~důrazem na nízkou spotřebu. Byl představen 
  v~roce 2010 jako součást specifikace Bluetooth 4.0 a je nekompatibilní s~původními verzemi. 
  Za vývoj a údržbu protokolu je odpovědná skupina Bluetooth SIG (Special Interest Group).
  V~současné době je nejnovější verzí Bluetooth 5 \cite{cesnet-survey}.
  
  \subsubsection{Způsob komunikace}
  BLE vysílá v~bezlicenčním ISM pásmu na frekvencích od 2.4\,GHz až do hodnoty 2.4835\,GHz. Specifikace
  protokolu vychází z~Bluetooth, ale zejména díky změnám parametrů v~rádiové vrstvě 
  jsou navzájem nekompatibilní. Do verze 4 umožňuje navázat pro koncová zařízení pouze jedno spojení,
  a proto 
  vytváří hvězdicovou topologii s~jedním centrálním prvkem. Od verze 5 je možné využívat více připojení
  a lze vytvořit flexibilnější mesh topologii \cite{cesnet-survey}. 
  
  Před zahájením komunikace mezi centrálním prvkem a senzory je nutné nejprve provést párování, 
  které probíhá v~následujících krocích:
  \begin{itemize}
   \item \textbf{Vysílání žádostí}
   
   Při spuštění párování začne koncový prvek na kanálech určených pro propagaci všesměrově vysílat žádosti o~připojení,
   které obsahují název zařízení, jméno výrobce a podporované služby \cite{ble-attack}.
   
   \item \textbf{Přijímání žádostí}
   
   Pokud je centrální prvek přepnutý do párovacího režimu, tak naslouchá příchozím požadavkům, které zobrazuje uživateli.
   Po výběru správného zařízení se ukončí mód naslouchání a začne se navazovat spojení \cite{ble-attack}.
   
   \item \textbf{Inicializace připojení} 
   
   Během této fáze se obě strany vzájemně domlouvají na parametrech komunikace \cite{ble-attack}.
   \item \textbf{Komunikace}
   
   Po úspěšném navázání spojení je možné na základě dostupných služeb provádět zasílání zpráv \cite{ble-attack}.
  \end{itemize}

 
   \subsubsection{Bezpečnost}
   BLE umožňuje volitelně používat šifrování pomocí AES, jehož šifrovací klíč je 
   vytvořen během párování v~části inicializace připojení. Velmi důležité je zabezpečit 
   výměnu sdíleného klíče, která až do verze 4.1 není bezpečná, protože neumožňuje ochranu
   před odposloucháváním. Vylepšení přichází až od verze 4.2, ve které lze využít ECDH \cite{cesnet-survey}.
   
   Vzhledem k~různorodosti možných typů zařízení definuje BLE následující kategorie
   určující způsob výměny klíče:
    \begin{itemize}
     \item \textbf{Just Works}
     
     Nejjednodušší třída, která provádí výměnu automaticky a zároveň má nejmenší nároky 
     na připojované zařízení, které nemusí podporovat autentizaci. Díky chybějící autentizaci 
     je tato metoda zranitelná vůči MITM útokům \cite{cesnet-survey}.
     \item \textbf{Out of Band}
     
     V~této kategorii jsou veškeré klíče vyměňovány odlišným komunikačním kanálem např. přes NFC (Near Field Communication).
     Celková bezpečnost této metody závisí na důvěryhodnosti použitého kanálu.
     \item \textbf{Passkey}
     
     Tato metoda vylepšuje \textit{Just Works} o~autentizaci, která spočívá v~uživatelském zadání 
     šestimístného kódu. Pro ochranu před odposlouchávání musí být použit algoritmus ECDH,
     který je dostupný až od verze 4.2~\cite{cesnet-survey}.
     
     \item \textbf{Numeric Comparison}
     
     Využití tohoto způsobu párování je možné pouze od verze 4.2. Dochází zde k~rozšíření metody \textit{Just Works}
     o~jeden kontrolní krok, který slouží jako ochrana před MITM útokem. Po výměně klíčů každé zařízení
     vygeneruje šestimístný číselný kód, který následně zobrazí uživateli a čeká na jeho potvrzení \cite{cesnet-survey}.   
     
    \end{itemize}

    
   
   \newpage
  \section{Bezpečnostní slabiny}
  Na základě provedené analýzy aktuálně používaných protokolů \ref{protokoly} budou v~této kapitole 
  popsány jejich bezpečnostní slabiny, které budou později využity při návrhu 
  detekčních algoritmů. Jelikož je tato práce zaměřena na senzorové protokoly 
  nekomunikující přes IP, budou v~podkapitolách popsány slabiny protokolů 
  Z-Wave a BLE.
  
 \subsection{Z-Wave}
 Hlavním problém jsou zařízení certifikovaná před březnem 2017, jelikož nepodporují S2 framework.
 Tyto prvky nemusí při své komunikaci využívat žádnou formu šifrování a síť se tak stává 
 velmi zranitelnou. V~minulosti již bylo provedeno několik útoků, které pomocí
 Scapy-Radio projektu \cite{ezwave} nebo knihovny OpenZWave  byly schopny ovládat jednotlivé senzory. 
 
 Při využití šifrování je velmi zranitelná doba během párovacího procesu, jelikož je při výměně
 šifrovacích klíčů použit výchozí klíč uložený ve firmware zařízení. Zároveň se 
 u~jednoho typu zámku podařilo objevit implementační chybu \cite{zwave-S0-attack}, která umožňovala vnutit 
 nový šifrovací klíč a převzít tak kontrolu nad senzorem. 
 
 Při využití S2 frameworku dosud nebyly objeveny žádné zranitelnosti, ovšem tento
 framework je poměrně nový a většina aktuálně používaných i nabízených zařízení byla certifikována 
 ještě před jeho zavedením.
 
 \subsection{BLE}
 Velkou hrozbou jsou zařízení s~verzí 4.1 nebo nižší, protože nepodporují žádnou ochranu před 
 odposloucháváním a MITM útokům v~době párování. Od verze 4.2 je již pro výměnu klíčů použit ECDH 
 algoritmus, který zabraňuje možnému odposlouchávání, ale v~případě použití párovací metody, které
 nepodporuje autentizaci není zajištěna ochrana před MITM útoky.  \cite{cesnet-survey}
 
 Dalším problémem jsou samotní výrobci, kteří často nevyužívají bezpečnostní funkce protokolu  \cite{ble-locks} nebo
 implementují vlastní způsoby zabezpečení na aplikační úrovni. Tímto dochází k~nezabezpečení fáze 
 párování a často se vyskytují i implementační chyby, které přinášení další zranitelnosti \cite{ble-attack} a
 umožňují útočníkovi získat kontrolu nad celým provozem.  \cite{cesnet-survey}
 
 Nevýhodou je velmi dlouhá a komplikovaná specifikace protokolu, která vede k~implementačním 
 chybám samotného BLE \cite{blueborne}. Tím vzniká nebezpečí, že i u~výrobce, který využívá
 všech bezpečnostní funkcionalit
 se mohou vyskytovat zranitelnosti díky chybné implementaci komunikačních vrstev protokolu. \cite{cesnet-survey}
 
 \newpage
 \section{Možnosti detekce}
 V~běžných IP sítích se pro detekci hrozeb nejčastěji používají IDS (Intrusion Detection System) a
 IPS (Intrustion Prevention System) systémy. Tyto služby rozšiřují koncept klasického firewallu, který
 blokuje nebo povoluje síťový provoz na základě statických pravidel, o~podrobnější sledování 
 datových toků, které jsou zablokovány na základě nestandardního chování.
 
 IDS/IPS může být reprezentováno samostatným hardwarovým zařízením nebo softwarovým programem, který
 může být dále rozšířen o~monitorovací sondy, které se starají pouze o~sběr dat a jejich odesílání
 pro následnou analýzu. Zároveň
 se může lišit i umístění v~síti. Detekční systémy lze provozovat před hraničním směrovačem 
 a detekovat tak kompletní příchozí a odchozí data nebo za hraničním směrovačem, což 
 umožní vyhodnocovat jen vyfiltrovaný provoz. Případně je možné nasadit IDS/IPS přímo na koncové
 stanice, kde kromě síťových dat lze získávat i informace o~běhu systému. Poslední způsob poskytuje
 nejvíce detailní možnost analýzy, protože se provádí na místě vzniku komunikace a případný útok
 je možné zastavit již při jeho začátku a zabránit tak případnému rozšíření. Nevýhodou takového nasazení
 je ztráta celkového pohledu na síť. Při zavedení IDS/IPS systému je z~těchto důvodů dobré 
 kombinovat způsoby nasazení a umožnit vzájemné sdílení detekovaných incidentů.
 
 Při zaměření na zpracovávání datových toků lze detekční systémy rozdělit do dvou základních kategorií:
 \begin{itemize}
  \item \textbf{Detekce anomálií}
  
  Tato metoda je založena na statistickém modelování. Nejprve se vytvoří profil běžného chování sítě, 
  který se následně porovnává s~aktuálním provozem. Dle použité metody se může profil běžného provozu
  průběžně aktualizovat. Pokud měřené údaje síťového provozu překročí hodnotu stanoveného profilu
  nad definovaný limit, tak je detekován incident. Výhodou metody detekce anomálií je její uplatnění i na 
  dosud neznámé útoky. Tento princip zároveň způsobuje větší míru falešných poplachů, a proto 
  je nutné jejich pečlivější ověření. Příkladem detekčních metod je: strojové učení, časové řady nebo
  stavové automaty \cite{ids-ips}.
  
  \item \textbf{Detekce signatur}
  
  V~případě použití této metody je využíváno signatur (profilů) předem známých útoků. Výhodou je, že díky 
  popsaným signaturám je tato metoda poměrně přesná a detekuje málo falešných poplachů. Naopak
  nevýhodou je, že není možné detekovat nové druhy útoků, pro které není známý profil. Problémem
  také je, že signatury musí být uloženy na nějakém perzistentním úložišti, které je dostupné 
  lokálně nebo vzdáleně \cite{ids-ips}.
 \end{itemize}
 
 IoT sítě k~běžnému IP provozu přidávají senzorové protokoly, jejichž chování je také dobré monitorovat, 
 protože jsou připojeny do počítačové sítě a mohou být zneužity při útocích. Bohužel detekční metody 
 nejsou pro tyto sítě moc rozšířené, a tím dochází ke zvýšení rizika a dopadu možných útoků. Pro určení 
 incidentů lze využít stejných principů jako v~IP sítích, ale liší se způsob získávání dat pro analýzu. 
 Pro sběr informací lze využít následující přístupy:
 \begin{itemize}
  \item \textbf{Testbed}
  
    Tato metoda spočívá ve vytvoření specializovaného prostředí, ve kterém se nachází pouze testované 
    a měřící zařízení. Cílem je ověřit, že nově připojovaný senzor splňuje veškeré bezpečnostní 
    požadavky a neobsahuje žádné známe zranitelnosti. Měřící prvky reprezentují nástroje, které 
    jsou schopné odposlouchávat komunikaci a využívají se také například při automatizovaných
    penetračních testech.   
  \item \textbf{Externí sonda}
  
  Funkce sondy je stejná jako v~IP sítích. Jedná se o~samostatné zařízení umístěné v~sítí, které
  umožňuje sledovat probíhající komunikaci a odesílat získané údaje ke zpracování. 
  Výhodou je velké množství různých  dat, které lze získat, ale zároveň komplikací je šifrovaný provoz
  a často i cena kvalitní sondy.
  Dalším využitím 
  může být honeypot, kdy se sonda tváří jako zranitelný prvek a reportuje veškeré pokusy o~útok.
    
  \item \textbf{Provozní statistiky}
  
  Posledním způsobem je sběr dat z~příslušných rozhraní na IoT bráně. Tento postup nevyžaduje použití
  žádného dalšího zařízení, ale potřebuje, aby brána umožňovala získávat tyto statistiky. 
  Statistiky nejsou tak podrobné jako u~externí sondy, ale výhodou je, 
  že získávání aktuálních dat o~provozu je poměrně nenáročné.
 \end{itemize}
 
 Každá z~předchozích metod používá jiný styl sběru dat. Při reálném použití je dobré vyhodnotit 
 bezpečnostní rizika a hrozby, podle kterých lze jednotlivé metody vhodně kombinovat.

 \newpage
 \section{Existující řešení}
 V~současnosti se veškerá dostupná řešení zaměřují na detekci útoků v~IP protokolech. Mezi 
 současnými IDS/IPS existují signatury pouze pro SCADA (Supervisory Control And Data Acquisition)
 protokoly. Router Turris Omnia umožňuje nasadit systém Suricata, pro který nabízí rozšíření 
 PaKon \cite{pakon}. Toto rozšíření zpracovává data ze Suricaty, které následně ukládá v~přehledné formě.
 Díky tomu uživatel získá podrobný přehled o~stavu provozu. Získaná data může dále využívat 
 například k~vylepšení stávajících bezpečnostních pravidel. Nevýhodou tohoto řešení jsou 
 vyšší hardwarové požadavky, a proto ho nelze provozovat na branách s~omezenými 
 prostředky. Další komplikací může být centralizovaná architektura a zaměření pouze na 
 IP provoz.
 
 Na základě provedené rešerše nebylo nalezeno žádné řešení, které umožňuje vyhodnocovat 
 provoz aktuálně používaných IoT protokolů s~ohledem na možné omezení hardwarových
 prostředků.
 
 \newpage
 \section{Analýza požadavků}
 Podstatnou částí návrhu výsledného řešení je přesné určení požadavků, které se dělí na 
 funkční a nefunkční. Funkční požadavky specifikují funkcionality kladené přímo na vznikající
 program, zatímco nefunkční spíše určují omezení vlastností systému a architekturu návrhu.
V~následující kapitole budou popsány nároky na vznikají detekční nástroj, které vycházejí z~obsahu
 zadání této práce a provedené analýzy.
 
  \subsection{Funkční požadavky}
  Kapitola popisuje funkční požadavky, které jsou od detekčního nástroje očekávány.
  \begin{itemize}
   \item \textbf{Sběr informací o~provozu}
   
   Program bude umožňovat sběr dat na IoT bráně o~aktuálním provozu z~dostupných komunikačních
   rozhraní. Zároveň bude možné získané informace přeposílat k další analýze.
   
   \item \textbf{Detekce anomálií}
   
   Vytvořený detekční algoritmus bude schopen na základě získaných dat odhalit definované anomálie,
   které reprezentují neočekávané změny v~síti. 
   
   \item \textbf{Konfigurace způsobu detekce}
   
   Detekční modul bude umožňovat nastavení parametrů pro jednotlivé zpracovávané položky
   pomocí konfiguračního souboru. Tyto 
   parametry budou následně sloužit jako vstup pro vytvořený detektor.
   
   \item \textbf{Zpracování získaných informací}
   
   Kromě analýzy dat a hlášení nalezených incidentů bude také možné zpracovaná data pravidelně 
   exportovat do dalších rozšiřujících modulů.
  \end{itemize}

  \subsection{Nefunkční požadavky}
  Kapitola obsahuje nefunkční požadavky, které jsou kladené na výsledný nástroj.
  \begin{itemize}
   \item \textbf{Rozšiřitelnost}
   
   Architektura celého řešení bude navržena tak, aby bylo možné rozšíření o~nové 
   způsoby detekce anomálií a další typy provozních dat.
   
   \item \textbf{Flexibilita nasazení}
   
   Návrh způsobu detekce umožní fyzicky oddělit komponentu zajišťující sběr dat a komponentu vyhodnocující 
   provoz. Díky tomu bude možné provádět analýzu i na IoT branách s~velmi omezeným prostředky, protože 
   tyto brány budou sloužit jako sondy, které budou posílat data do externího detektoru s~dostatečnými 
   výkonem k~provedení analýzy. 
   
   \item \textbf{Kompatibilita a vývoj}
   
   Pro zachování kompatibility a snadného vývoje bude použit framework NEMEA
    \cite{nemea} (Network Measurements Analysis), který umožní snadné propojení 
    jednotlivých detekčních modulů.
   
   \item \textbf{Operační systém}
   
   Výsledné řešení bude implementováno a otestováno na operačním systému Ubuntu 16.04, na kterém 
   bude nasazena IoT brána BeeeOn a framework NEMEA. Zároveň
   budou při návrhu a implementaci vybírány takové technologie, aby bylo možné vytvořený program
   spustit pod distribuci OpenWrt, které je velmi rozšířena mezi síťovými prvky.   
   
  \end{itemize}

 \newpage 
 \section{Zvolené řešení}
 Obsahem této kapitoly je popis zvoleného řešení, které bylo na základě provedené analýzy
 určeno pro realizaci výsledného nástroje.
 
 Vytvořený detekční algoritmus bude zaměřen na senzorové IoT protokoly, protože v~současné době 
 neexistuje řešení, které by to umožňovalo. Pro analýzu budou použity protokoly Z-Wave a BLE, 
 které jsou v~dnešních sítích velmi rozšířené. Algoritmus bude umístěn přímo na IoT bráně, která 
 je ideálním místem, protože
 se nachází velmi blízko koncových zařízení a má dostatečný výkon k~vyhodnocování provozu. 
 Konkrétně bude použita brána BeeeOn, protože v~současné době 
 jako jediná umožňuje sběr provozních dat o~senzorových protokolech. 
 
 Pro účely detekce budou sbírány jen informace dostupné z~lokálních rozhraní brány.
 Je velice pravděpodobné, že z~hlediska reálného nasazení bude tato varianta nejčastější, 
 protože nevyžaduje žádná dodatečná monitorovací zařízení.
 Zároveň bude umožněno rozšíření i pro další
 způsoby získávání dat. 
 
 Vyhodnocování dat bude probíhat metodou detekce anomálií, která je vhodná pro statistickou povahu dat,
 má menší nároky na množství dostupných prostředků a umožňuje rozpoznat i neznámé útoky. Metoda 
 bude realizovaná pomocí časových řad jejichž parametry bude možné upravit dle potřeby.
 
 Jako programovací jazyk bude použit C++, protože podporuje objektový přístup a
 úspornou implementaci na paměť i procesorový čas. Zároveň tento jazyk je použit i v~bráně
 BeeeOn, tudíž bude usnadněna integrace. Pro zajištění flexibility a možnosti dalšího rozšiřování 
 bude využito systému NEMEA, který také podporuje jazyk C++. 



\chapter{Návrh}
Kapitola se zabývá návrhem způsobu detekce anomálií v~IoT sítích, který s~pomocí NEMEA
systému rozšiřuje bránu BeeeOn. Nejprve je popsána celková architektura a
možné způsoby nasazení. Dále následuje popis jednotlivých komponent detekčního systému.
V~závěru jsou identifikovány možné scénáře útoků.

   \section{Možnosti detekce}
  % testbed, externi sonda, provozni data
 \section{Možnosti nasazení}
  % celkova architektura reseni, lokalni&externi detekce
  % microservice architektura, mozne rozsireni o IP exporter 
 \section{Scénáře útoků}
 \section{Kolektor}
 \section{Detektor}
 \section{Multiplexor a demultiplexor}


\chapter{Realizace}
Obsahem kapitoly je popis realizace nejzajímavějších částí vytvořeného programu. Pro zajištění
efektivity a snadné přenositelnosti byla veškeré implementace provedena v~jazyce C++, ve kterém
je zároveň napsána BeeeOn brána. Detekční systém přináší pouze jednu novou závislost a tou je NEMEA framework.
Ostatní použité knihovny se shodují s~již použitými v~BeeeOn bráně.

 
\section{Integrace kolektoru}    
     % popsat trochu jak probihala integrace kolektoru
     Vytvořený kolektor provozních dat je přímou součástí projektu BeeeOn, a proto jej bylo nutné 
     integrovat do spouštěcího procesu brány. Pro zajištění společné kompilace byly do souborů 
     \textit{CMakeLists.txt} přidány cesty ke zdrojovým souborům kolektoru a závislosti na knihovny
     z NEMEA frameworku. K určení způsobu spouštění jednotlivých komponent používá BeeeOn
     soubor \textit{factory.xml}. V tomto souboru byla vytvořena nová komponenta s názvem 
     \textit{collector}, 
     která byla následně přidána pod označením \textit{listener} do ostatních komponent. Toto nastavení 
     umožňuje přijímání definovaných událostí v rámci návrhového vzoru \textit{Observer}.
     Samotná komponenta
     obsahuje ve svém popisu seznam jmen událostí, kterým přiřazuje pojmenování výstupního \textit{libtrap}
     rozhraní. Formát výstupu je vždy \textit{event-<názevUdálosti>}. 
     
     Vytvořené názvy událostí jsou při spouštění programu pomocí C++ reflexe předány třídě kolektoru
     \textit{NemeaCollector}. Tato třída používá již vytvořené makro \textit{BEEEON\_OBJECT\_TEXT},
     které pro definovaný seznam 
     událostí volá příslušné členské funkce. Každá událost má členskou funkci s názvem ve formátu 
     \textit{set<názevUdálosti>} přijímající jeden vstupní parametr typu string s názvem výstupního 
     rozhraní. V rámci volání se do instance třídy \textit{EventMetaData}
     nastaví specifické hodnoty členských
     atributů dané události. Následuje volání členské funkce \textit{initInterface(EventMetaData)},
     která přijímá vytvořenou instanci třídy  \textit{EventMetaData} a
     jednotně inicializuje všechny potřebné struktury pro odesílání dat.
     
     Po úspěšném nastavení všech částí se už jen v rámci návrhového vzoru \textit{Observer} volají
     členské funkce událostí, které jsou definované ve třídě \textit{AbstractCollector} a implementované
     ve třídě \textit{NemeaCollector}. Získané informace jsou vkládány do UniRec zprávy a odeslány
     výstupním rozhraním.
     
\section{Mux a Demux}    

 Modul \textit{Mux} očekává na vstupu přepínač \textit{-i}, který ve formě řetězce určuje dostupná 
 rozhraní 
 zajištěné knihovnou \textit{libtrap}. Poslední identifikátor v řetězci označuje jméno 
 výstupu. Druhým parametrem je \textit{-n}, který odpovídá počtu vstupních rozhraní. Při spuštění 
 se pro každý vstup pomocí knihovny OpenMP vytvoří samostatné vlákno. Jelikož je veškerý provoz 
 odesílaný jedním společným rozhraním, bylo nutné funkci \textit{trap\_ctx\_send} vložit do
 kritické sekce, 
 protože pracuje se sdílenými strukturami pro všechny vlákna.
 
 Společné spojení mezi moduly vytvořenými \textit{Mux} a \textit{Demux} používá nastavený typ
 \textit{TRAP\_FMT\_RAW},
 který umožňuje posílat zprávy ve vlastně definovaném formátu. Hlavička vytvořeného formátu
 obsahuje: identifikátor druhu zprávy, číslo rozhraní a typ formátu. Obsah přijatých zpráv
 je zapouzdřen do záhlaví. \textit{Mux} při každém přijetí dat kontroluje návratový
 kód funkce \textit{trap\_ctx\_recv}, který identifikuje nový formát přijatých zpráv. Pokud
 dojde k detekování změny, tak se pošle \textit{hello} zpráva s upraveným popisem rozhraní.
 V ostatních případech se jen přeposílají zapouzdřená data.
 
 Modul \textit{Demux} vyžaduje stejné přepínače jako \textit{Mux}. Jediným rozdílem je, že 
 název společného rozhraní, které má \textit{Mux} na posledním místě, musí být zde uveden 
 jako první. Důvodem je, že knihovna \textit{libtrap} zpracovává nejprve vstupní a pak
 výstupní rozhraní. 
 
\section{Zpracování zadaných parametrů}

Načtení konfiguračního souboru má na starosi třída \textit{ConfigParser}, která neprovádí 
žádnou kontrolu vstupních dat, protože se již od návrhu předpokládá, že konfigurace bude 
generována odlišným programem, který zajistí správnost parametrů. Jediný konstruktor třídy
 \textit{ConfigParser} očekává jako parametr řetězec s cestu k cílovému souboru. Během vytváření
 objektu jsou postupně zpracovány jednotlivé řádky zadaného souboru. Zároveň budou incializovány
 pole pro uchovávání vypočtených profilů. Jejich délka je specifikována preprocesorovou direktivou
 \textit{\#define DYNAMIC}.
 
Pokud byly zadány parametry pro pravidelný export dat, tak při inicializaci potřebných struktur
zavolá funkce s názvem \textit{initExportInterfaces}, která vytvoří příslušná výstupní 
rozhraní. Jejich název je vždy vygenerován v následujícím formátu: 
\textit{u:export-<názevKlíče><idKlíče>}.
   
\section{Výpočet profilu}   
   %popis vypoctu profilu
\section{Detekční funkce} 
 % algoritmus detektoru popsat jednotlivé detekční metody
  % popsat, ze periodicky veci jsou delany pomoci vlaken per polozku
    % hlavní je že je to proto, že se nedozvi info o čase když nepřijdou data, tak to musim dělat sám

\chapter{Testování}
V~této kapitole je popsán postup testování vytvořeného detekčního řešení. Nejprve je představeno
testovací prostředí a jsou otestovány obě možné metody nasazení. Dále jsou ověřeny jednotlivé 
případy hrozeb z~kapitoly \ref{utoky} a také jsou prověřeny ostatní funkcionality
detektoru. V~záměru je provedeno měření chování jednotlivých položek z~profilu.

 
\section{Testovací prostředí}
Veškeré provedené testy proběhly na lokálním počítači s~operačním systémem Ubuntu 16.04,
na kterém zároveň proběhl vývoj nástroje. Systémové
prostředky dostupné pro testovaní byly: 4 jádra CPU, 8 GB RAM a 17 GB SSD disk.

Na počítači byl nasazen systém NEMEA a IoT brána BeeeOn, která byla rozšířena o~vytvořený detekční 
nástroj. Pro generování senzorových dat byly použity následující zařízení: BLE teplotní senzor (BeeWi), 
Z-Wave zásuvky (POPP a Fibaro) a virtuální senzory, které jsou dostupné pro testování
v~rámci BeeeOn brány.
Senzorová data přijímal testovací počítač, který měl připojený USB Z-Wave dongl (Aeotec) a integrované
BLE rozhraní.

\section{Způsoby nasazení}
Jako první byly testovány možnosti nasazení vytvořeného nástroje, který je možné používat
v~následujících režimech:

 \begin{itemize}
  \item \textbf{Lokální režim}
  
  V~lokální variantě byl použit pouze modul detektoru a kolektoru. Aby se nemusela pro každou 
  událost spouštět nová instance detektoru, byl využit již existující NEMEA modul \textit{Merger},
  který 
  dokáže jednotlivé UniRec zprávy spojit do jedné konsolidované. Způsob nasazení je zobrazen na Obrázku 
  \ref{obr.option1}.
  
  \begin{figure}[ht]
   \begin{center}
   \includegraphics[scale=0.5]{pictures/deploy-option1}
   \caption{Lokální nasazení}
   \label{obr.option1}
   \end{center}
   \end{figure}
   
   Kolektor vždy odesílá dostupné události výstupními komunikačními rozhraními
   dle konfigurace brány.
   Exportovaná data přijímá NEMEA modul \textit{Merger}, který je spojuje a předává detektoru.
   
  \item \textbf{Oddělený režim} \label{externalMode}
  
  Druhý způsob nasazení byl také otestován na lokálním zařízení, protože lze využít lokálních
  soketů dostupných přes NEMEA framework. Provedení realizace je na Obrázku \ref{obr.option2}.
 
  \begin{figure}[ht]
   \begin{center}
   \includegraphics[scale=0.5]{pictures/deploy-option2}
   \caption{Oddělené nasazení}
   \label{obr.option2}
   \end{center}
   \end{figure}
   
   Složení komponent vychází z~prvního případu, který byl rozšířen o~moduly Mux a Demux, které 
   umí spojit a rozdělit přicházející komunikaci.
 \end{itemize}
 
Cílem testů bylo, aby detektor obdržel všechny odeslané zprávy z~kolektoru. Ověření bylo provedeno
spuštěním detektoru s~přepínačem \textit{-vv}, který zapne ladící výpisy druhé úrovně pro zobrazení 
přijatých UniRec polí. Výsledky obou případů byly úspěšné a všechna data byla přijata.

\section{Detekce scénářů útoků} \label{testAttack}
Velmi důležitou částí bylo otestování definovaných anomálií, které mohou reprezentovat skutečný
útok na sít. Tato sekce popisuje testy pro jednotlivé scénáře.

  \begin{enumerate}
    \item \textbf{Periodicita dat}
    
    Tento scénář nebylo nutné dělit na jednotlivé protokoly, protože popisuje obecné chování 
    připojených senzorů bez ohledu na technologii. Cílem bylo odhalit provoz, který porušuje
    očekávaný periodický průběh. V~rámci detekce nebyly potřebné naučené profily sítě, 
    ale využívalo se parametrů ze skupiny \textit{general} umožňujících periodické kontroly.
    
    První test byl určený na odhalení nepravidelného přijetí dat. Pro ověření byla použita událost
    \textit{onExport}, která poskytuje hodnoty získané ze senzorů, a v~konfiguračním souboru byla 
    aktivována pravidelná kontrola dat každých 8 sekund.
    
    Výsledek testu byl úspěšný. Pokud nebyla obdržena žádná data déle než 8 sekund, tak byly
    odeslány informace o~incidentu.
    
    Několik událostí poskytovaných branou BeeeOn jsou čistě periodické a~po uplynutí definovaného
    času vždy odešlou dostupné statistiky. Druhý test byl zaměřen na odhalení případu, kdy se 
    přijímané datové položky nemění, což může reprezentovat odpojení nebo ztrátu čidla. Pro 
    otestování byla zvolena událost \textit{onHciStats}, která získává informace o~provozu BLE sítě. 
    Dále byla v~konfiguračním souboru nastavena pravidelná kontrola dat na 7 sekund a maximálně 
    5 po sobě přijatých hodnot mohlo být stejných. 
    
    Výsledek byl pozitivní, protože při přijetí více než pěti stejných po sobě jdoucích
    hodnot byla detekována anomálie.
    
    \item \textbf{Množství přenášených zpráv}
    
    V~případě protokolu Z-Wave byly pro nasimulování anomálií použity dvě vzdáleně ovladatelné
    zásuvky. Obě mají k~dispozici ovládací tlačítko pro vypnutí a zapnutí zásuvky. Tímto způsobem
    byly generovány nové zprávy. Cílem tohoto scénáře bylo odhalit neočekávaný nárůst dat, a~proto
    byla zvolena detekční metoda, která hlídá tyto změny. V~rámci testu byly do profilu 
    vloženy všechny dostupné položky. Limitem pro ohlášení incidentu byl pětinásobný nárůst
    provozu. Délka časové řady byla nastavena na deset prvků a prvních jedenáct přijatých 
    zpráv bylo ignorováno, protože během nich bylo navazováno spojení.
    
    První detekce byla zaměřena na položku \textit{SOAFCount}, která určuje celkový počet 
    detekovaných zpráv.  
    Druhý scénář sledoval hodnotu \textit{receivedCount}, která reprezentuje počet přijatých zpráv od
    konkrétního čidla. 
    
    Pro BLE byla časová řada zkrácena na pět prvků, žádné zprávy nebyly ignorovány a limitem
    pro určení anomálie byl dvojnásobný nárůst dat.
    Test byl proveden pro položku \textit{rxBytes}, která určuje množství obdržených zpráv.
    Údaje byly vyčítány skriptem, který v~čase měnil četnost zaslaných požadavků o~data.
    
    Výsledky testů byly úspěšné a každá změna provozu byla odhalena. Průběh všech testovaných
    případů byl velmi podobný a lišil se jen typ události a způsob generování dat. Chování 
    jednotlivých položek profilu velmi ovlivňovala délka časové řady, která určovala 
    paměťové okno. Nejcitlivější položkou na změnu byl rozptyl, který výrazně zvyšoval
    svou hodnotu při vložení vyššího čísla. K~častým výchylkám docházelo i v~případě průměru. 
    Méně frekventované změny nastávaly u~mediánu, který nejvíce využíval délky časové řady. 
    Nejmenších odlišností dosahoval klouzavý průměr, protože ve své hodnotě obsahuje i data
    mimo aktuální časové okno.

    \item \textbf{Limity senzorových hodnot}
    
    Pro tento scénář byla použita data vygenerována virtuálními senzory, které jsou dostupné
    v~rámci brány BeeeOn. Do konfiguračního profilu byly vloženy všechny položky, pro které
    byly specifikovány parametry pro očekávané soft a hard limity. Cílem scénáře bylo odhalení
    změn v~aktuálním profilu provozu, které porušují předepsané limity. 
    
    Výsledek detekce splnil očekávání a nepovolené změny byly úspěšně detekovány. Chování 
    jednotlivých položek profilu se shodovalo s~předchozím scénářem.
    
    \item \textbf{Kvalita přenosového kanálu}
    
    Cílem tohoto případu užití bylo detekování změny kvality přenosového kanálu. Pro Z-Wave jsou 
    k~tomuto účelu vhodné položky \textit{lastResponseRTT} a \textit{dropped}. Údaj
    \textit{lastResponseRTT} je součástí události \textit{onNodeStats} a 
    \textit{dropped} je obsažen v~\textit{onDriverStats}. Z~tohoto důvodu byl při generování dat použit
    NEMEA modul \textit{Merger}, který spojil dvě různě události do jedné. Zároveň bylo nutné
    v~konfiguračním souboru
     popsat obě hodnoty. V~případě \textit{lastResponseRTT} byla jako anomálie označena
    hodnota klouzavého průměru časové řady přesahující dvojnásobek vzorového profilu a nebo byla nižší
    než polovina stanoveného klouzavého průměru. Pro položku \textit{dropped} bylo jako incident považováno
    libovolné zvětšení klouzavého mediánu. Proto byla velikost časového okna rovna jedné a hard limitu byl
    nastaven na jedna. 
    
    Jelikož došlo ke spojení dvou různých událostí, kde \textit{onNodeStats}
    umožňuje získat informace pro jednotlivé senzory a \textit{onDriverStats} pro celou síť, 
    tak výsledkem byla událost poskytující informace pro každý koncový prvek. Z~tohoto důvodu 
    musel být v~konfiguraci uveden v~rámci klíče identifikátor příslušných zařízení určených k~analýze.
    
    Výstupy detekce se shodovaly s~předpoklady, a proto byl výsledek testu úspěšný. V~současné
    době se zatím nepodařilo připravit zkušební prostředí, ve které by bylo možné otestovat
    rušení kanálu, a tím ovlivnit hodnotu \textit{lastResponseRTT}. Pro účely otestování 
    detekce však byla postačující ruční úprava dat v~souboru se zachyceným provozem. 
    
    V~případě BLE by se využily položky \textit{rxErrors} nebo \textit{txErrors} a nastavení 
    detekce by bylo shodné s~\textit{lastResponseRTT}. Vzhledem k~tomu, že by došlo jen 
    ke změně názvu klíče v~konfiguračním souboru, tento test nebyl proveden.
    
    \item \textbf{Konektivita}
    
    Pro ověření posledního scénáře lze využít poznatků z~předchozích testů. Ztráta konektivity
    může být způsobena neobdržením očekávané události a to lze detekovat pomocí periodicity dat.
    Druhým způsobem je výrazné zhoršení statistik přenosového kanálu, které byly zpracovány
    v~předchozím případu užití.
  \end{enumerate}
  
  Ověření definovaných scénářů umožnilo otestování správné funkcionality vytvořeného detekčního
  systému a zároveň byla sestavena množina informací, kterou je nutné sledovat k~získání 
  přehledu o~stavu sítě.
  
  Vygenerovaný provoz, na kterém byly jednotlivé případy užití otestovány je uložen i s~nalezenými
  anomáliemi a konfiguračním souborem na přiloženém CD. V~případě testu periodicity dat nebylo možné záznam 
  komunikace uložit, protože použitá detekce využívá aktuální čas. Z~tohoto důvodu 
  byl do souboru místo záznamu uložen popis, jak takový provoz vygenerovat.

\section{Test exportu dat}
Identifikované scénáře útoků otestovaly většinu funkcí detekčního systému. Vytvořený nástroj
ovšem poskytuje i možnost pravidelného exportu definovaných položek aktuálního profilu, a tím 
vytváří informace pro další pokročilejší detekce. 

Sada testů ověřila, že vypočítané části profilu lze spolehlivě exportovat. Výsledek byl tedy
pozitivní.

\section{Měření položek profilu}
Již v~kapitole \ref{testAttack} bylo možné sledovat odlišné chování 
jednotlivých částí profilu. V~této sekci je provedeno vyhodnocení každé z~nich. Veškeré 
měření probíhalo nad stejným provozem, kde byly zprávy generovány každé tři sekundy a~pravidelné
události chodily každých pět sekund. Incidenty byly reprezentovány nárůstem sledovaného provozu.

Tabulka \ref{tab.tab1} zobrazuje pro každou položku profilu počet detekovaných anomálií 
s~rostoucí velikostí časové řady. Její vyšší rozměry nejvíce stabilizovaly klouzavý medián, protože jeho počty
incidentů klesly se zvyšující se délkou. Zároveň z~jeho grafu \ref{obr.progressMedian} lze vidět,
že začátek anomálií odhalil vždy jako poslední a jejich doba trvání byla nejkratší. 

S~větší časovou řadou bylo možné uchovat i více hodnot, a 
tím se detekovalo více událostí v~rámci klouzavého průměru a rozptylu. Začátky i konce rozpoznání incidentů
byly u~obou položek velmi podobné. Jediným rozdílem u~klouzavého rozptylu bylo, že při stabilním 
nárůstu sledované položky může jeho hodnota po uplynutí časového okna opět klesnout do bezpečného pásma. 
Graf průběhu \ref{obr.progressStandardDeviation} pro větší přehlednost nezobrazuje klouzavý rozptyl, ale 
klouzavou směrodatnou odchylku. 

Průměr jako jediný nebyl závislý na časovém okně. Ovšem s~rostoucí délkou řady 
byla delší i doba učení, při které byla vždy stanovena lehce odlišná hodnota vzorového
profilu a mezních limitů. Při provedeném měření se vždy s~delším paměťovým oknem vzorové hodnoty mírně 
zvýšily.
Na základě těchto předpokladů lze v~jeho grafu \ref{obr.progressCumAverage}
vidět pomalé klesání, díky kterým byly incidenty detekovány nejdelší dobu. Zároveň dle očekávání velikost 
časového okna nemá vliv na počet detekovaných anomálií.



\begin{table}[ht]
  \begin{center}
  \begin{tabular}{|C{1.7cm}|C{1.7cm}|C{1.7cm}|C{1.7cm}|C{1.7cm}|}
    \hline 
   \thead{velikost\\ časového\\ okna} & \thead{klouzavý\\ medián} & \thead{klouzavý\\ rozptyl} & \thead{klouzavý\\ průměr} & \thead{průměr}\\
   \hline 
   \hline 
    1 & 26 & 0 & 26 & 26\\
    \hline
    3 & 26 & 30 & 30 & 90\\
    \hline
    5 & 26 & 34 & 32 & 89\\
    \hline
    10 &  26 & 44 & 40 & 89\\
    \hline
    15 & 19 & 54 & 48 & 89\\
    \hline
    20 & 19 & 62 & 56 & 89\\
    \hline
   \end{tabular}
   \caption{Tabulka obsahuje počty detekovaných anomálií s~rostoucí délkou časového okna} 
   \label{tab.tab1}
  \end{center}   
    \end{table}

    \begin{figure}[ht]
   \begin{center}
   \includegraphics[scale=0.7]{pictures/moving_median_progress}
   \caption{Graf vývoje klouzavého mediánu s~časovou řadou délky 10}
   \label{obr.progressMedian}
   \end{center}
   \end{figure}
   
   \begin{figure}[ht]
   \begin{center}
   \includegraphics[scale=0.7]{pictures/moving_standard_deviation_progress}
   \caption{Graf vývoje klouzavé směrodatné odchylky s~časovou řadou o délce 10}
   \label{obr.progressStandardDeviation}
   \end{center}
   \end{figure}
    
  \begin{figure}[ht]
   \begin{center}
   \includegraphics[scale=0.7]{pictures/moving_average_progress}
   \caption{Graf vývoje klouzavého průměru s~časovou řadou délky 10}
   \label{obr.progressAverage}
   \end{center}
   \end{figure}
   
   \begin{figure}[ht]
   \begin{center}
   \includegraphics[scale=0.7]{pictures/average_progress}
   \caption{Graf vývoje průměru s~časovou řadou délky 10}
   \label{obr.progressCumAverage}
   \end{center}
   \end{figure}

\begin{conclusion}
	%sem napište závěr Vaší práce
\end{conclusion}

\bibliographystyle{csn690}
\bibliography{mybibliographyfile}

\appendix

\chapter{Seznam použitých zkratek}
% \printglossaries
\begin{description}
	\item[IoT] Internet of Things
	\item[IP] Internet Protocol
	\item[HTTPS] Hypertext Transfer Protocol Secure  
	\item[VPN] Virtual Private Network 
	\item[MQTT] Message Queuing Telemetry Transport 
	\item[CoAP] Constrained Application Protocol 
	\item[AMQP] Advanced Message Queuing Protocol
	\item[DDoS] Distributed Denial Of Service
	\item[MITM] Man In The Middle
	\item[M2M] Machine To Machine 
	\item[TCP] Transmission Control Protocol
	\item[QoS] Quality of Service
	\item[TLS] Transport Layer Security
	\item[RESTful] Representational State Transfer
	\item[UDP] User Datagram Protocol
	\item[HTTP] Hypertext Transfer Protocol
	\item[URI] Uniform resource identifier
	\item[DTLS] Datagram Transport Layer Security
	\item[ISM] Industrial, Scientific and Medical
	\item[AES] Advanced Encryption Standard
	\item[PIN] Personal Identification Number
	\item[QR] Quick Response
	\item[ECDH] Elliptic-curve Diffie–Hellman
	\item[BLE] Bluetooth Low Energy 
	\item[SIG] Special Interest Group
	\item[NFC] Near Field Communication
	\item[IDS] Intrusion Detection System
	\item[IPS] Intrustion Prevention System
	\item[SCADA] Supervisory Control And Data Acquisition
	\item[NEMEA] Network Measurements Analysis
	\item[UniRec] Unified Record
	\item[RAII] Resource Acquisition Is Initialization
	\item[RTT] Round-Trip Time
\end{description}

\chapter{Instalační příručka}
Realizovaný detekční systém pro svůj běh využívá framework NEMEA a projekt BeeeOn. V následujících
krocích je popsána instalace těchto dvou nástrojů, způsob nasazení kolektoru a
spuštění detekčních modulů.

\begin{enumerate}
 \item \textbf{Instalace projektu BeeeOn}
 
 Zdrojové kódy tohoto projektu se nachází v repozitáři služby Github. Pro jeho sestavení je nutné
 vykonat následující příkazy: 
\begin{verbatim}
git clone https://github.com/BeeeOn/gateway.git --recursive
cd  gateway
mkdir build
(cd build && cmake ..)
make -C build
\end{verbatim}
 \item \textbf{Instalace NEMEA frameworku}
 
Kompletní zdrojové kódy systému NEMEA jsou také udržované v repozitáři služby Github. Pro potřeby
vytvořeného nástroje je potřeba část \textit{NEMEA-Framework}.
\begin{verbatim}
git clone https://github.com/CESNET/Nemea-Framework.git
cd Nemea-Framework
./bootstrap.sh
./configure
make
sudo make install
\end{verbatim}

Pro spouštění vytvořených testů nebo pro úpravu příchozích a odchozích dat ve formátu UniRec se
mohou hodit moduly \textit{Logger}, \textit{Logreplay} a \textit{Merger}, které se nacházejí
v částí \textit{NEMEA-Modules}. V tomto 
repozitáři se zároveň nachází vytvořené moduly \textit{Mux} a \textit{Demux}.
\begin{verbatim}
git clone https://github.com/CESNET/Nemea-Modules.git
cd Nemea-Modules
./bootstrap.sh
./configure
make
sudo make install
\end{verbatim}

\item \textbf{Nasazení kolektoru}

V rámci tohoto kroku je potřeba zdrojové kódy vytvořeného kolekotru přesunout do BeeeOn brány
a upravit konfigurační soubory. Postup je popsán v následujících sekcích.
\begin{itemize}
 \item \textbf{Přesun zdrojových souborů}
\begin{verbatim}
   cp NemeaCollector.* <adresářBeeeOn>/src/core
   cp fields.* <adresářBeeeOn>/src/core
\end{verbatim}
\item \textbf{Úprava souborů pro kompilaci}
\begin{verbatim}
   vim <adresářBeeeOn>/src/CMakeLists.txt
\end{verbatim}
Do příslušných částí v souboru je potřeba vložit tyto položky: 
\begin{verbatim}
   find_library (LIBTRAP trap)
   find_library (UNIREC unirec)
   find_library (PCAP pcap)

   ${PROJECT_SOURCE_DIR}/core/fields.cpp
   ${PROJECT_SOURCE_DIR}/core/NemeaCollector.cpp

   ${PCAP}
   ${UNIREC}
   ${LIBTRAP}
\end{verbatim}
Podobné změny je nutné provést v adresářit \textit{test}
\begin{verbatim}
  vim <adresářBeeeOn>/test/CMakeLists.txt
    find_library (LIBTRAP trap)
    find_library (UNIREC unirec)
    find_library (PCAP pcap)
    
    ${PCAP}
    ${UNIREC}
    ${LIBTRAP} 
\end{verbatim}
 \item \textbf{Úprava sestavovací konfigurace brány}
\begin{verbatim}
   vim <adresářBeeeOn>/conf/config.d/factory.xml 
\end{verbatim}
V otevřeném souboru je nutné do odpovídajících elementů vložit tyto parametry:
\begin{verbatim}
   #element instance name="distributor"
   <add name="listener" ref="collector"/>

   #element instance name="bluetoothAvailability"
   <add name="listeners" ref="collector"/>

   #element instance name="zwaveDeviceManager"
   <add name="listeners" ref="collector"/>

   #element instance name="commandDispatcher"
   <add name="listeners" ref="collector"/>

   #element factory
   <instance name="collector" 
             class="BeeeOn::NemeaCollector">
     <set name="onExportInterface" 
          text="u:event-onExport"/>
     <set name="onHCIStatsInterface" 
          text="u:event-onHCIStats"/>
     <set name="onDispatchInterface" 
          text="u:event-onDispatch"/>
     <set name="onNodeStatsInterface" 
          text="u:event-onNodeStats"/>
     <set name="onDriverStatsInterface" 
          text="u:event-onDriverStats"/>
   </instance> 
\end{verbatim} 
\item \textbf{Rekompilace brány}
\begin{verbatim}
   cd <beeeonRootDir>
   (cd build && cmake ..)
   make -C build
\end{verbatim}
\end{itemize}

\item \textbf{Kompilace detektoru}

Kompilace komponenty detektoru se provede následujícím příkazem:
\begin{verbatim}
g++ DataDetector.cpp fields.cpp ConfigParser.cpp \
Analyzer.cpp -o prg -ltrap -lpcap -lunirec --std=c++11 \
-Wno-write-strings -pthread 
\end{verbatim}

\item \textbf{Spuštění detekčního systému}

Posledním krokem je spuštění celého detekčního systému. Pro účely demonstrace je
v ukázce použit oddělený model nasazení \ref{externalMode}, který zahrnuje více komponent. Při reálném 
použití stačí spustit pouze kolektor a jednu instanci detektoru pro každý typ události.
\begin{verbatim}
<adresářBeeeOn>/build/src/beeeon-gateway \
   -c conf/gateway-testing.ini -Dtesting.center.enable=yes 
mux "u:event-onDriverStats,u:event-onNodeStasts,u:output" \
   -n 2
demux "u:output,u:onDriverStats,u:onNodeStats" -n 2
merger "u:onDriverStats,u:onNodeStats,u:merged" -n 2
<adresářDetector>/prg -i "u:merged,u:alert" -v
logger "u:alert"
logger "u:export-dropped0"
\end{verbatim}
\end{enumerate}


\chapter{Nástroje pro testování}
V následující kapitole přílohy se nachází popis způsobu použití nástrojů, které byly uplatněny 
v rámci testování.

\begin{itemize}
 \item \textbf{Logger}

 Tento NEMEA modul slouží k ukládání přijatých UniRec zpráv. Při spuštění je nutné zadat 
 přepínač \textit{-i} s popisem vstupního komunikačního rozhraní. Dále je dobré použít přepínač
 \textit{-w <názevSouboru>} pro ukládání přijatých dat do souboru, \textit{-t} pro uložení datové
 hlavičky příchozích
 dat a \textit{-T}, který ukládá i časové značky.
 
 Příklad spuštění: \textit{logger -t -T -i u:alert -w alert-data.log}
 
 \item \textbf{Logreplay}
 
 Modul \textit{Logreplay} lze využít pro přehrání zachyceného provozu pomocí modulu \textit{Logger}.
 Mezi povinné
 přepínače patří
 \textit{-i}, za kterým následuje popis výstupního komunikačního rozhraní, a \textit{-f <názevSoboru>}
 pro načtení
 uloženého provozu. Vhodným přepínačem může být \textit{-n}, které neposílá po přehrání EOF (End Of File)
 zprávu. Pokud soubor se zachycenými daty obsahuje časové značky každého záznamu, tak jsou 
 jednotlivé zprávy odesílány dle těchto značek.
 
 Ukázkou použití: \textit{logreplay -i u:zwave -f z-wave-connection.log -n}
 
 \item \textbf{Merger}
 
 Poslední uvedený NEMEA modul se používá pro slučování několika různcýh vstupních UniRec
 záznamů do jednoho společného, který je odesílán výstupním rozhraním. Očekávánými parametry jsou
 \textit{-i} s popisem komunikačního rozhraní a \textit{-n} určující počet vstupních rozhraní,
 který odpovídá
 zadanému popisu. Název výstupu je vždy uveden jako poslední položka v rámci přepínače \textit{-i}.
 
 Možné zavedení: \textit{merger u:DriverStasts,u:NodeStats,u:merged -n 2}
 

\end{itemize}


\chapter{Popis množiny senzorových informací} \label{sensorData}
Tato část přílohy obsahuje kompletní popis množiny informací, kterou lze v současné době získat o 
provozu senzorových protokolů v rámci BeeeOn brány. Při popisu byla množina informací rozdělena
do příslušných událostí, které
jsou reprezentovány samostatnou sekcí.

\section{onDriverStats}
\begin{description}
 \item \textbf{SOAFCount}: počet přijatých Start Of Frame bytů
 \item \textbf{ACKWaiting}: počet nevyžádaných zpráv při čekání na potvrzení
 \item \textbf{readAborts}: počet nedokončených operacích čtení z důvodu překročení časového limitu
 \item \textbf{badChecksum}: počet zpráv se špatných kontrolním součtem
 \item \textbf{readCount}: počet úspěšně přijatých zpráv
 \item \textbf{writeCount}: počet úspěšně odeslaných zpráv
 \item \textbf{CANCount}: počet přijatých CAN bytů
 \item \textbf{NAKCount}: počet přijatých negativních potvrzení
 \item \textbf{ACKCount}: počet přijatých potvrzení
 \item \textbf{OOFCount}: počet přijatých Out Of Frame bytů
 \item \textbf{dropped}: počet zahozených zpráv 
 \item \textbf{retries}: počet znovu odeslaných zpráv
 \item \textbf{callbacks}: počet neočekávaných callbacků
 \item \textbf{badroutes} počet neodeslaných zpráv kvůli směrování
 \item \textbf{noACK}: počet neobržených potvrzení
 \item \textbf{netBusy}: počet zpráv s chybovým stavem
 \item \textbf{notIdle}: počet zpráv se stavem not idle
 \item \textbf{nonDelivery}:počet nedoručených zpráv
 \item \textbf{routedBusy}:  počet přijatých zpráv se stavem routed busy
 \item \textbf{broadcastReadCount}: počet přijatých všesměrových zpráv
 \item \textbf{broadcastWriteCount}:počet odeslaných všesměrových zpráv
\end{description}

\section{onNodeStats}
\begin{description}
 \item \textbf{sentCount}: počet odeslaných zpráv
 \item \textbf{sentFailed}: počet zpráv, které se nepodařilo odeslat
 \item \textbf{receivedCount}: počet přijatých zpráv
 \item \textbf{receivedDuplications}: počet přijatých duplikovaných zpráv
 \item \textbf{receivedUnsoliced}: počet přijatých nevyžádaných zpráv
 \item \textbf{lastRequestRTT}: RTT poslední odeslané zprávy
 \item \textbf{lastResponseRTT}: RTT poslední přijaté odpovědi
 \item \textbf{averageRequestRTT}: průměr z lastRequestRTT
 \item \textbf{averageResponseRTT}: průměr z lastResponseRTT
 \item \textbf{quality}: kvalita připojení
 \item \textbf{nodeID}: identifikátor senzoru
\end{description}

\section{onHCIStats}
\begin{description}
 \item \textbf{address}: adresa BLE rozhraní 
 \item \textbf{aclMtu}: velikost MTU pro ACL zprávy
 \item \textbf{aclPackets}: velikost zásobníku pro ACL zprávy 
 \item \textbf{scoMtu}: velikost MTU pro SCO zprávy
 \item \textbf{scoPackets}: velikost zásobníku pro SCO zprávy
 \item \textbf{rxErrors}: počet přijatých chybových zpráv
 \item \textbf{txErrors}: počet odeslaných chybových zpráv
 \item \textbf{rxEvents}: počet přijatých BLE událostí
 \item \textbf{txCmds}: počet odeslaných BLE příkazů
 \item \textbf{rxAcls}: počet přijatých ACL zpráv
 \item \textbf{txAcls}: počet odeslaných ACL zpráv
 \item \textbf{rxScos}: počet přijatých SCO zpráv
 \item \textbf{txScos}: počet odeslaných SCO zpráv
 \item \textbf{rxBytes}: počet přijatých bytů
 \item \textbf{txBytes}: počet odeslaných bytů
\end{description}

\section{onExport}
\begin{description}
 \item \textbf{value}: senzorová hodnota
 \item  \textbf{deviceID}: intentifikátor senzoru
\end{description}


% % % % % % % % % % % % % % % % % % % % % % % % % % % % 
% % Tuto kapitolu z výsledné práce ODSTRAŇTE.
% % % % % % % % % % % % % % % % % % % % % % % % % % % % 
% 
% \chapter{Návod k~použití této šablony}
% 
% Tento dokument slouží jako základ pro napsání závěrečné práce na Fakultě informačních technologií ČVUT v~Praze.
% 
% \section{Výběr základu}
% 
% Vyberte si šablonu podle druhu práce (bakalářská, diplomová), jazyka (čeština, angličtina) a kódování (ASCII, \mbox{UTF-8}, \mbox{ISO-8859-2} neboli latin2 a nebo \mbox{Windows-1250}). 
% 
% V~české variantě naleznete šablony v~souborech pojmenovaných ve formátu práce\_kódování.tex. Typ může být:
% \begin{description}
% 	\item[BP] bakalářská práce,
% 	\item[DP] diplomová (magisterská) práce.
% \end{description}
% Kódování, ve kterém chcete psát, může být:
% \begin{description}
% 	\item[UTF-8] kódování Unicode,
% 	\item[ISO-8859-2] latin2,
% 	\item[Windows-1250] znaková sada 1250 Windows.
% \end{description}
% V~případě nejistoty ohledně kódování doporučujeme následující postup:
% \begin{enumerate}
% 	\item Otevřete šablony pro kódování UTF-8 v~editoru prostého textu, který chcete pro psaní práce použít -- pokud můžete texty s~diakritikou normálně přečíst, použijte tuto šablonu.
% 	\item V~opačném případě postupujte dále podle toho, jaký operační systém používáte:
% 	\begin{itemize}
% 		\item v~případě Windows použijte šablonu pro kódování \mbox{Windows-1250},
% 		\item jinak zkuste použít šablonu pro kódování \mbox{ISO-8859-2}.
% 	\end{itemize}
% \end{enumerate}
% 
% 
% V~anglické variantě jsou šablony pojmenované podle typu práce, možnosti jsou:
% \begin{description}
% 	\item[bachelors] bakalářská práce,
% 	\item[masters] diplomová (magisterská) práce.
% \end{description}
% 
% \section{Použití šablony}
% 
% Šablona je určena pro zpracování systémem \LaTeXe{}. Text je možné psát v~textovém editoru jako prostý text, lze však také využít specializovaný editor pro \LaTeX{}, např. Kile.
% 
% Pro získání tisknutelného výstupu z~takto vytvořeného souboru použijte příkaz \verb|pdflatex|, kterému předáte cestu k~souboru jako parametr. Vhodný editor pro \LaTeX{} toto udělá za Vás. \verb|pdfcslatex| ani \verb|cslatex| \emph{nebudou} s~těmito šablonami fungovat.
% 
% Více informací o~použití systému \LaTeX{} najdete např. v~\cite{wikilatex}.
% 
% \subsection{Typografie}
% 
% Při psaní dodržujte typografické konvence zvoleného jazyka. České \uv{uvozovky} zapisujte použitím příkazu \verb|\uv|, kterému v~parametru předáte text, jenž má být v~uvozovkách. Anglické otevírací uvozovky se v~\LaTeX{}u zadávají jako dva zpětné apostrofy, uzavírací uvozovky jako dva apostrofy. Často chybně uváděný symbol "{} (palce) nemá s~uvozovkami nic společného.
% 
% Dále je třeba zabránit zalomení řádky mezi některými slovy, v~češtině např. za jednopísmennými předložkami a spojkami (vyjma \uv{a}). To docílíte vložením pružné nezalomitelné mezery -- znakem \texttt{\textasciitilde}. V~tomto případě to není třeba dělat ručně, lze použít program \verb|vlna|.
% 
% Více o~typografii viz \cite{kobltypo}.
% 
% \subsection{Obrázky}
% 
% Pro umožnění vkládání obrázků je vhodné použít balíček \verb|graphicx|, samotné vložení se provede příkazem \verb|\includegraphics|. Takto je možné vkládat obrázky ve formátu PDF, PNG a JPEG jestliže používáte pdf\LaTeX{} nebo ve formátu EPS jestliže používáte \LaTeX{}. Doporučujeme preferovat vektorové obrázky před rastrovými (vyjma fotografií).
% 
% \subsubsection{Získání vhodného formátu}
% 
% Pro získání vektorových formátů PDF nebo EPS z~jiných lze použít některý z~vektorových grafických editorů. Pro převod rastrového obrázku na vektorový lze použít rasterizaci, kterou mnohé editory zvládají (např. Inkscape). Pro konverze lze použít též nástroje pro dávkové zpracování běžně dodávané s~\LaTeX{}em, např. \verb|epstopdf|.
% 
% \subsubsection{Plovoucí prostředí}
% 
% Příkazem \verb|\includegraphics| lze obrázky vkládat přímo, doporučujeme však použít plovoucí prostředí, konkrétně \verb|figure|. Například obrázek \ref{fig:float} byl vložen tímto způsobem. Vůbec přitom nevadí, když je obrázek umístěn jinde, než bylo původně zamýšleno -- je tomu tak hlavně kvůli dodržení typografických konvencí. Namísto vynucování konkrétní pozice obrázku doporučujeme používat odkazování z~textu (dvojice příkazů \verb|\label| a \verb|\ref|).
% 
% \begin{figure}\centering
% 	\includegraphics[width=0.5\textwidth, angle=30]{cvut-logo-bw}
% 	\caption[Příklad obrázku]{Ukázkový obrázek v~plovoucím prostředí}\label{fig:float}
% \end{figure}
% 
% \subsubsection{Verze obrázků}
% 
% % Gnuplot BW i barevně
% Může se hodit mít více verzí stejného obrázku, např. pro barevný či černobílý tisk a nebo pro prezentaci. S~pomocí některých nástrojů na generování grafiky je to snadné.
% 
% Máte-li například graf vytvořený v programu Gnuplot, můžete jeho černobílou variantu (viz obr. \ref{fig:gnuplot-bw}) vytvořit parametrem \verb|monochrome dashed| příkazu \verb|set term|. Barevnou variantu (viz obr. \ref{fig:gnuplot-col}) vhodnou na prezentace lze vytvořit parametrem \verb|colour solid|.
% 
% \begin{figure}\centering
% 	\includegraphics{gnuplot-bw}
% 	\caption{Černobílá varianta obrázku generovaného programem Gnuplot}\label{fig:gnuplot-bw}
% \end{figure}
% 
% \begin{figure}\centering
% 	\includegraphics{gnuplot-col}
% 	\caption{Barevná varianta obrázku generovaného programem Gnuplot}\label{fig:gnuplot-col}
% \end{figure}
% 
% 
% \subsection{Tabulky}
% 
% Tabulky lze zadávat různě, např. v~prostředí \verb|tabular|, avšak pro jejich vkládání platí to samé, co pro obrázky -- použijte plovoucí prostředí, v~tomto případě \verb|table|. Například tabulka \ref{tab:matematika} byla vložena tímto způsobem.
% 
% \begin{table}\centering
% 	\caption[Příklad tabulky]{Zadávání matematiky}\label{tab:matematika}
% 	\begin{tabular}{|l|l|c|c|}\hline
% 		Typ		& Prostředí		& \LaTeX{}ovská zkratka	& \TeX{}ovská zkratka	\tabularnewline \hline \hline
% 		Text		& \verb|math|		& \verb|\(...\)|	& \verb|$...$|		\tabularnewline \hline
% 		Displayed	& \verb|displaymath|	& \verb|\[...\]|	& \verb|$$...$$|	\tabularnewline \hline
% 	\end{tabular}
% \end{table}
% 
% % % % % % % % % % % % % % % % % % % % % % % % % % % % 

\chapter{Obsah přiloženého CD}

%upravte podle skutecnosti

\begin{figure}
	\dirtree{%
		.1 readme.txt\DTcomment{stručný popis obsahu CD}.
		.1 exe\DTcomment{adresář se spustitelnou formou implementace}.
		.1 src.
		.2 impl\DTcomment{zdrojové kódy implementace}.
		.2 thesis\DTcomment{zdrojová forma práce ve formátu \LaTeX{}}.
		.1 text\DTcomment{text práce}.
		.2 thesis.pdf\DTcomment{text práce ve formátu PDF}.
		.2 thesis.ps\DTcomment{text práce ve formátu PS}.
	}
\end{figure}

\end{document}
